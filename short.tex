% Root File for a UCSB Dissertation
\documentclass[12pt,oneside,final]{ucthesis}

\usepackage{array}
\usepackage{cite}
\usepackage{color}
\usepackage{graphicx}
\usepackage[breaklinks]{hyperref}
\usepackage{import}
\usepackage[super]{nth}
\usepackage{times}
\usepackage{url}

% http://tex.stackexchange.com/questions/73112/how-to-create-local-labels-refs-in-latex
% macro to define a local label
\newcommand\locallabel[1]{\label{\currentprefix:#1}}
% macro to use a local reference
\newcommand\localref[1]{\ref{\currentprefix:#1}}

% For Hairball chapter
\newcommand{\abs}{\emph{absolute}}
\newcommand{\broadcast}{\emph{broadcast EVENT}}
\newcommand{\correct}{\emph{correct}}
\newcommand{\greenflag}{\emph{when green flag clicked}}
\newcommand{\incom}{\emph{incomplete}}
\newcommand{\incor}{\emph{incorrect}}
\newcommand{\initzone}{\emph{initialization zone}}
\newcommand{\playsound}{\emph{play SOUND}}
\newcommand{\playsounddone}{\emph{play SOUND until done}}
\newcommand{\receive}{\emph{when I receive EVENT}}
\newcommand{\rel}{\emph{relative}}
\newcommand{\say}{\emph{say MESSAGE}}
\newcommand{\sayfor}{\emph{say MESSAGE for SECONDS}}
\newcommand{\semincor}{\emph{semantically incorrect}}
% For Curriculum chapter
\newcommand{\sprogram}{Scratch program}
\newcommand{\sone}{\emph{Sequential1}}
\newcommand{\stwo}{\emph{Sequential2}}
\newcommand{\catch}[1]{\emph{catch#1}}
\newcommand{\caught}{\emph{caught}}
\newcommand{\com}{\emph{complete}}
\newcommand{\dce}{\emph{double click to execute}}
\newcommand{\exe}[1]{execution sequence#1}
\newcommand{\glideDIST}[1][NUM]{\emph{glide #1 steps}}
\newcommand{\glideto}[1][SPRITE]{\emph{glide to #1}}
\newcommand{\pointtoward}[1][SPRITE]{\emph{point towards #1}}
\newcommand{\pointDIR}[1][X]{\emph{point in direction #1}}
\newcommand{\netclicked}{\emph{when NET clicked}}
\newcommand{\bear}{\textbf{Bear}}
\newcommand{\horse}{\textbf{Horse}}
\newcommand{\net}{\textbf{Net}}
\newcommand{\snake}{\textbf{Snake}}
\newcommand{\zebra}{\textbf{Zebra}}
\newcommand{\stage}{\emph{stage}}
% For Feedback Chapter
\newcommand{\AM}[1]{#1AM}
\newcommand{\PM}[1]{#1PM}
\newcommand{\cw}[2][A]{\emph{#1W#2}}
\newcommand{\cs}[2][A]{\emph{#1S#2}}
\newcommand{\cm}[2][A]{\emph{#1M#2}}
\newcommand{\cf}[2][A]{\emph{#1F#2}}
\newcommand{\imp}{\emph{Improvement}}
\newcommand{\worse}{\emph{Worse}}
\newcommand{\noi}{\emph{No Improvement}}
\newcommand{\noii}{\emph{No Improvement 2}}
\newcommand{\spacco}[1][.\@]{Spacco et al#1}

\newif\iffull
\fullfalse


%%% Document Portion:

\begin{document}

%% Front Matter:
%%%%%%%%%%%%%%%%%%%%%%%%%%%
% TITLE PAGE INFORMATION %
%%%%%%%%%%%%%%%%%%%%%%%%%%%

\title{Enabling Wide-Scale Computer Science Education through Improved
                Automated Assessment Tools}

\author{Bryce A. Boe}

%%%%%%%%%%%%%%%%%%%%%%%%%%%%%%%%%%
% DECLARATIONS FOR FRONT MATTER %
%%%%%%%%%%%%%%%%%%%%%%%%%%%%%%%%%%
\report{Dissertation}
\degree{Doctor of Philosophy}
\degreemonth{September}
\degreeyear{2014}
\approvalmonth{September}
\approvalyear{2014}

\chair{Dr. Diana Franklin, LSOE}
\committeeII{Professor Timothy Sherwood}
\committeeIII{Professor Danielle Harlow}
\nummembers{3}

\field{Computer Science}
\campus{Santa Barbara}

\begin{frontmatter}
%\begin{abstract}

\addcontentsline{toc}{chapter}{Abstract}

Enrollment in computer science undergraduate programs has increased nationally
since 2008. Meanwhile the demand for college graduates to fill computer science
related jobs continues to increase. In order to produce more computer science
graduates, computer science departments must increase their enrollment
sizes. In lieu of hiring additional faculty, many departments are investigating
methods for scaling the number of students. Despite the growing increase in
demand for computer science professionals, computer science at the primary
school level is virtually nonexistent. As a result, significant effort is being
made to incorporate computational thinking into existing primary school
education. Validating and deploying new curriculum across the nation is a great
challenge.

To enable wide-scale computer science education we do two things. First, we
created a framework to support the static analysis of Scratch and Scratch-like
projects, named Hairball. Scratch is a popular building-block language utilized
to pique interest in, and teach the basics of computer science. We describe how
we utilize Hairball to verify and test that our learning objectives are met in
our pilot computational thinking curriculum targeted for fourth, fifth and
sixth graders. The use of Hairball allows for rapid curriculum alterations and
thus contributes to wide-scale deployment of \nth{4} -- \nth{6} grade computer
science curriculum.

Second, we created a real-time feedback and assessment system that has been
utilized extensively by students of two UCSB courses since winter quarter
2013. The configuration of the system was systematically varied from assignment
to assignment and between instances of the courses to measure changes in
student submission behavior. We show that changes in the system configuration
affect student submission behavior. This system, along with its findings, has a
positive impact on supporting wide-scale computer science education.

\end{abstract}

\end{frontmatter}


% the chapters

% spacing in figures and tables and their captions can be
% changed here (\ssp for single-space, empty for same as surrounding
% text); for this to work, the command \figsp has to be included
% in every figure and table right after the \begin{figure}
\def\figsp{\ssp}
%\def\figsp{}


\chapter{Introduction} \label{chap:introduction}
A recent report by code.org suggests that by 2020 there will be a one million
person gap in the United States between the number of vacant computer science
jobs and the number of computer scientists available to fill these
jobs~\cite{codeorg:2013}. The lack of newly trained computer scientists is a
twofold problem:

\begin{itemize}
\item There are not enough students interested in computer science coming out
  of high school, evidenced by the fact that of the 2.1 million students
  nationwide who took AP exams in 2013, only 31,000 (1.4\%) students took the
  AP Computer Science exam~\cite{cb:2013}. The low numbers of high school
  students studying computer science could be due to the absence of
  availability of computer science-related instruction in areas such as
  computational thinking and elementary programming in primary through
  secondary curricula. The Computer Science Teachers Association acknowledges
  this problem in their 2011 K--12 Computer Science Standards where they detail
  the path to resolution requiring the incorporation of computer science
  concepts beginning in primary school~\cite{cstastandards}.
\item At this point in time, the university system does not have the resources
  to adequately handle the rapid, yet insufficient, increase in numbers of
  students applying to computer science programs. The future does not look much
  more encouraging, as departments may not be able to support computer science
  enrollment growth through an increase in resources including faculty
  size. For example, Lazowska et al.\ report that while the student body of
  both Princeton and MIT comprises more than 10\% computer science majors, it
  is unlikely that 10\% of the total university faculty will ever be part of
  computer science~\cite{lazowska:2014}. The unfortunate result is that
  university computer science departments are turning away a significant number
  of qualified students from a discipline where they are severely needed.
\end{itemize}

This dissertation describes methods I developed along with my colleagues that
have a positive impact on solving both the challenge to get more students
interested in computer science coming out of high school, as well as the
challenge to support the growth in number of university-level computer science
students. My work focuses on ways to help increase student interest in computer
science by introducing a \nth{4}--\nth{6} grade Scratch-based computational
thinking curriculum by supporting rapid curriculum development via a
design-based research approach. The upcoming wide-scale deployment of this
curriculum in California will reach more than a thousand young students in the
next year alone, with more reach in subsequent years. My work also investigates
the important issue of how to increase the number of university-level computer
science students through the incorporation of a real-time feedback and
assessment system into existing computer science curriculum. My goal is for the
system to reduce the amount of time that instructors currently devote to the
labor-intensive assessment process, thereby making more time available to spend
with students who need extra assistance. My research, in combination with
future efforts, has the promise to enable the university system to produce more
computer scientists.

\section{Thesis Statement}
The increase in interest in computer science has resulted in a need to scale
computer science instruction from the primary school grades through
undergraduate level university programs. These two ends of the spectrum,
however, are in far different places in their development, with very little
curriculum existing for primary schools and very mature instruction available
at the university level. Assessment automation can greatly enhance both efforts
by allowing us to understand certain important aspects of student learning
behavior. At the primary school level, assessment automation through static
analysis of student work can provide instructors with insight into student
comprehension, enabling rapid curriculum changes that result in faster
deployment of new curriculum. At the university level, automated feedback and
assessment systems provide large numbers of students with immediate insight
into their success with class assignments allowing them to iteratively achieve
mastery of course topics, and reduce assessment time permitting instructors to
focus their efforts on students in need.

\section{Dissertation Overview}
The remainder of this dissertation is structured as follows. To promote young
students’ interest in computer science --- and hopefully their continued
interest through secondary levels --- I investigate the use of static analysis
in the curriculum development and assessment processes for \nth{4}--\nth{8}
grade students that focuses on computational thinking and introductory
programming. Specifically, in Chapter~\ref{chap:hairball} I look at the use of
static analysis to assist with the \emph{post assessment} of five Scratch
assignments given in a two-week Scratch-based \emph{summer camp} for
\nth{6}--\nth{8} grade students. In Chapter~\ref{chap:curriculum} I extend the
use of static analysis of Scratch assignments to aid in the \emph{development}
of a \nth{4}--\nth{6} grade \emph{classroom-based computational thinking
  curriculum}. In an effort to support increasing numbers of computer science
students in university level classes, in Chapter~\ref{chap:feedback} I look at
\emph{submission behaviors} of undergraduate computer science students in the
presence of my real-time feedback and assessment system. Finally, in
Chapter~\ref{chap:conclusion} I summarize my findings and discuss the impact of
my research on computer science education at both the primary and university
instructional levels.

%\subimport{chap2/}{hairball}
%\subimport{chap3/}{curriculum}
%\subimport{chap4/}{feedback}
%\chapter{Conclusion}
TODO: Write conclusion.


\bibliography{resources/references}
\bibliographystyle{abbrv}%{unsrt}%{alpha}{siam}{apalike}{ieeetr}{plain}{acm}{abbrv}

\end{document}
