% Root File for a UCSB Dissertation
\documentclass[12pt,oneside,final]{ucthesis}

\usepackage{array}
\usepackage{cite}
\usepackage{color}
\usepackage{graphicx}
\usepackage[breaklinks]{hyperref}
\usepackage{import}
\usepackage[super]{nth}
\usepackage{times}
\usepackage{url}

% http://tex.stackexchange.com/questions/73112/how-to-create-local-labels-refs-in-latex
% macro to define a local label
\newcommand\locallabel[1]{\label{\currentprefix:#1}}
% macro to use a local reference
\newcommand\localref[1]{\ref{\currentprefix:#1}}

% For Hairball chapter
\newcommand{\abs}{\emph{absolute}}
\newcommand{\broadcast}{\emph{broadcast EVENT}}
\newcommand{\correct}{\emph{correct}}
\newcommand{\greenflag}{\emph{when green flag clicked}}
\newcommand{\incom}{\emph{incomplete}}
\newcommand{\incor}{\emph{incorrect}}
\newcommand{\initzone}{\emph{initialization zone}}
\newcommand{\playsound}{\emph{play SOUND}}
\newcommand{\playsounddone}{\emph{play SOUND until done}}
\newcommand{\receive}{\emph{when I receive EVENT}}
\newcommand{\rel}{\emph{relative}}
\newcommand{\say}{\emph{say MESSAGE}}
\newcommand{\sayfor}{\emph{say MESSAGE for SECONDS}}
\newcommand{\semincor}{\emph{semantically incorrect}}
% For Curriculum chapter
\newcommand{\sprogram}{Scratch program}
\newcommand{\sone}{\emph{Sequential1}}
\newcommand{\stwo}{\emph{Sequential2}}
\newcommand{\catch}[1]{\emph{catch#1}}
\newcommand{\caught}{\emph{caught}}
\newcommand{\com}{\emph{complete}}
\newcommand{\dce}{\emph{double click to execute}}
\newcommand{\exe}[1]{execution sequence#1}
\newcommand{\glideDIST}[1][NUM]{\emph{glide #1 steps}}
\newcommand{\glideto}[1][SPRITE]{\emph{glide to #1}}
\newcommand{\pointtoward}[1][SPRITE]{\emph{point towards #1}}
\newcommand{\pointDIR}[1][X]{\emph{point in direction #1}}
\newcommand{\netclicked}{\emph{when NET clicked}}
\newcommand{\bear}{\textbf{Bear}}
\newcommand{\horse}{\textbf{Horse}}
\newcommand{\net}{\textbf{Net}}
\newcommand{\snake}{\textbf{Snake}}
\newcommand{\zebra}{\textbf{Zebra}}
\newcommand{\stage}{\emph{stage}}
% For Feedback Chapter
\newcommand{\AM}[1]{#1AM}
\newcommand{\PM}[1]{#1PM}
\newcommand{\cw}[2][A]{\emph{#1W#2}}
\newcommand{\cs}[2][A]{\emph{#1S#2}}
\newcommand{\cm}[2][A]{\emph{#1M#2}}
\newcommand{\cf}[2][A]{\emph{#1F#2}}
\newcommand{\imp}{\emph{Improvement}}
\newcommand{\worse}{\emph{Worse}}
\newcommand{\noi}{\emph{No Improvement}}
\newcommand{\noii}{\emph{No Improvement 2}}
\newcommand{\spacco}{Spacco et al.}

\newif\iffull
\fullfalse


%%% Document Portion:

\begin{document}

%% Front Matter:
%%%%%%%%%%%%%%%%%%%%%%%%%%%
% TITLE PAGE INFORMATION %
%%%%%%%%%%%%%%%%%%%%%%%%%%%

\title{Enabling Wide Scale Computer Science Education through Improved
                Automated Assessment Tools}

\author{Bryce Boe}

%%%%%%%%%%%%%%%%%%%%%%%%%%%%%%%%%%
% DECLARATIONS FOR FRONT MATTER %
%%%%%%%%%%%%%%%%%%%%%%%%%%%%%%%%%%
\report{Dissertation}
\degree{Doctor of Philosophy}
\degreemonth{September}
\degreeyear{2014}
\approvalmonth{September}
\approvalyear{2014}

\chair{Dr. Diana Franklin, LSOE}
\committeeII{Professor Timothy Sherwood}
\committeeIII{Professor Danielle Harlow}
\nummembers{3}

\field{Computer Science}
\campus{Santa Barbara}

\begin{frontmatter}
%
%  Abstract
%

\begin{abstract}

\addcontentsline{toc}{chapter}{Abstract}

TODO: Provide the dissertation abstract.

\end{abstract}

\end{frontmatter}


% the chapters

% spacing in figures and tables and their captions can be
% changed here (\ssp for single-space, empty for same as surrounding
% text); for this to work, the command \figsp has to be included
% in every figure and table right after the \begin{figure}
\def\figsp{\ssp}
%\def\figsp{}


\chapter{Introduction}
TODO: Provide introduction to dissertation that lays out the next three
sections.

\section{Motivation and Background}
TODO: Provide the motivation and background.

\section{Thesis Statement}
The increase in popularity of Computer Science has resulted in large demand for
Computer Science instruction, from primary school through college. These two
ends of the spectrum, however, are in far different places in their
development, with very little curriculum existing for primary schools and very
mature instruction available for college. Assessment automation can greatly
enhance both efforts, albeit in different ways. At the primary school level,
assessment automation through static analysis of student work can provide
insight to student comprehension thus permitting rapid curriculum changes, but
this insight is limited by the structure of assignments and the amount of data
collected. At the collegiate level, automated assessment can provide students
with insight to their success on an assignment allowing them to iteratively
achieve mastery, however, the timeliness and quantity of feedback may inhibit
students' mastery of important secondary processes.

\section{Dissertation Overview}
TODO: Provide the dissertation overview.

\subimport{chap2/}{hairball}
\subimport{chap3/}{curriculum}
\subimport{chap4/}{feedback}
\chapter{Conclusion} \label{chap:conclusion}

In this dissertation I have demonstrated the effectiveness of static analysis
both in the post-assessment of a Scratch-based summer camp, and in the
development of a \nth{4} -- \nth{6} grade Scratch curriculum. And I reported on
the submission behavior of college students in the presence of a real-time
feedback and assessment system. The significance of this collective research is
to support both the growth in number of students who seek computer science
education, and to do so while maximizing student performance.

I envision a future, where years from now, many students who will have
completed our primary school curriculum will face a decision to choose a
college major. Many of these students will select a major involving some degree
of computation thinking. Their assignments will be electronically submitted,
and specifically designed to provide them the optimal amount of feedback at the
optimal time to maximize their understanding in the minimum amount of time
enabling students of the future to learn a significant amount more in a shorter
amount of time.

These potential advancements in student learning are only possible due to
analysis and assessment of student in-progress work that occur via studies
similar to mine. Every bit of information expands our general knowledge of
student learning. Only through the iterative application and subsequent
measurement of a new or altered technique can computer science education
advance to the point where this future becomes reality. I hope we all see this
future.


\bibliography{resources/references}
\bibliographystyle{abbrv}%{unsrt}%{alpha}{siam}{apalike}{ieeetr}{plain}{acm}{abbrv}

\end{document}
