\section{Results}\locallabel{sec:results}
In this section we describe the results of our analysis of the \sone{} and
\stwo{} data. In general we wanted to gain insight into the following
questions:

\begin{itemize}
\item How successful were students in creating a \sprogram{} that completed the
  assignment?
\item Did the changes we made after \sone{} improve student completion rates?
\item How pervasive were the challenges identified by researchers via direct
  student observation?
\end{itemize}

To answer these questions we (Section~\localref{sub:by_class}) look at the
completion rate of students by class, (Section~\localref{sub:snapshots})
compare the difficulty of \sone{} and \stwo{} based on the number of snapshots
to completion, (Section~\localref{sub:approach})analyze the approach students
used in solving the assignment, (Section~\localref{sub:race}) quantify the
number of students who may have experienced a race condition in Scratch, and
(Section~\localref{sub:dce}) quantify the number of students who may have
utilized the \dce{} approach when initially working on the assignment.


\subsection{Students by Class}
\locallabel{sub:by_class}

\begin{figure}[!t]
\centering
\includegraphics[width=3.3in]{graphs/by_class_students.eps}
\caption{Compares the maximum number of sprites \caught{} by student by
  class. A student is considered \com{} if any of their snapshots \catch{es}
  two or more sprites.}
\locallabel{fig:by_class_students}
\end{figure}

We analyzed data from seven of the ten classes listed in
Table~\localref{table:classes}, two for \sone{}, and five for \stwo{}. This
data include a total of 297 snapshots, twenty-nine students for \sone{}, and
638 snapshots, 120 students for \stwo{}. We have more data for \stwo{} due to
having more participating classes, all of which contained more consent-giving
students.

The total height of each bar in Figure~\localref{fig:by_class_students}
indicates the number of students by class; this number is displayed above each
bar. The different colored portions of each bar groups students by the maximum
number of sprites \caught{}. Green, pink, and yellow respectively indicate that
all, two, or only one of the three sprites were \caught{}. Purple indicates
none of those students' snapshots result in the \net{} \catch{ing} a sprite
upon execution. The four students in the \emph{0 Sprites} group are interesting
because the base-project provided to all students \catch{es} one sprite, the
\zebra{}, on execution. Thus, these four students made changes resulting in
negative progress toward the goal. Also of note is that of the 102 students who
\caught{} at least two sprites, only twelve did not \catch{} the final sprite.

We determine the success of a snapshot by running it through a Hairball plugin
that emulates the \net{}'s movement according to the \net{}'s \emph{script}
beginning with the \netclicked{} block. A snapshot is considered \com{} if the
emulated movement of the \net{} results in intersection with any two or more
sprites corresponding to the \bear{}, \horse{}, and \zebra{}. In the event
intersection with the \snake{} occurs (only valid for \sone{}), the snapshot is
considered \incom{}. Of the 297 \sone{} snapshots, only one was \incom{} due to
intersection with the \snake{}. A student is considered \com{} if they have at
least one \com{} snapshot.

\subsection{Number of Snapshots to Completion}
\locallabel{sub:snapshots}

\begin{figure}[!t]
\centering \includegraphics[width=3.3in]{graphs/snapshots_to_completion.eps}
\caption{Depicts the percentage of students by class that completed the
  assignment by the number of snapshots as indicated on the x-axis. The dashed
  cyan line representing \emph{S1A} was truncated; it would otherwise extend
  horizontally out to the twenty-first snapshot.}
\locallabel{fig:snapshots_to_completion}
\end{figure}

In the previous section we looked at the total number of sprites \caught{} by
students in the seven classes analyzed. While this information provides us with
the overall completion rate, it does not provide any insight related to the
difficulty of the assignment. We approximate the assignment difficulty for a
student as the number of snapshots saved up to their first \com{}
snapshot. Recall from Section~\localref{sub:collection} that we consider each
snapshot to be a unit of work.

Figure~\localref{fig:snapshots_to_completion} plots the number of snapshots
saved for students in each class on their path to completion. An increase in
the \emph{y-value} for a line indicates what percent more students were able to
\com{} the assignment after the corresponding number of snapshots. The end of a
line indicates the maximum number of snapshots generated on the path to
completion for students of that class. This figure clearly conveys two
discrepancies between \sone{}, indicated by dashed-lines, and \stwo{},
indicated by solid-lines:

\begin{itemize}
\item All \stwo{} classes, save for \emph{S2B}, had a higher completion rate
  than the two \sone{} classes.
\item More importantly, this figure shows that \stwo{} was considerably less
  difficult to complete than \sone{} based on the strictly fewer number of
  snapshots to completion for all \stwo{} classes, again save for \emph{S2B}.
\end{itemize}

Over 50\% of \stwo{} students completed by snapshot three, whereas fewer than
25\% of \sone{} students completed by that snapshot. Furthermore, approximately
20\% of \sone{} students completed after nine snapshots. \stwo{} was less
challenging to the students due in part to the addition of the \glideto{}
block. We look specifically at the impact of the \glideto{} block in the next
section.

\subsection{Approach to Solving the Assignment}
\locallabel{sub:approach}
As previously described, this assignment asks students to program a set of
directions to navigate the \net{} to \catch{} the \bear{}, the \horse{},
and the \zebra{}. This set of directions can be constructed in a number of
ways. At the highest level, there are two approaches:

\textbf{Glide to SPRITE}: With a single \glideto{} block the \net{} will glide
on a direct path to the target sprite resulting in an intersection between
two. The simplest \com{} solution requires only three of these blocks, one for
each of the \bear{}, the \horse{}, and the \zebra{}. This approach was only
available in \stwo{}.

\textbf{Orient and Glide}: The other high-level approach is to modify the
\net{}'s orientation via one of three classes of orientation changing blocks,
and then to glide an appropriate number of steps via a \glideDIST{} block in
order to reach the desired target or waypoint. The three classes of orientation
changing blocks are:

\textbf{Absolute orientation}: This orientation change is accomplished via a
\pointDIR{} block where \emph{X} can be selected as \emph{up (0)}, \emph{right
  (90)}, \emph{down (-180)}, or \emph{left (-90)}. Alternatively, any number
can be manually entered for a more precise orientation. These orientations are
absolute with respect to the \stage{} meaning \emph{up} always orients toward
the top of the \stage{}.

\textbf{Relative orientation}: This orientation change is accomplished via
either a \emph{turn clockwise NUM degrees}, or a \emph{turn counterclockwise
  NUM degrees} block. The use of one of these blocks results in a modification
to the current orientation of the \net{}. That is, if the \net{} is oriented
toward the right of the \stage{}, a \emph{turn clockwise 90 degrees} block will
result in the \net{} being oriented toward the bottom of the \stage{}; in this
case, the \abs{} orientation block \pointDIR[down]{} would have the same
effect.

\textbf{Sprite orientation}: The third class of orientation changing blocks is
accomplished via a \pointtoward{} block. When invoked as
\pointtoward[\zebra{}]{}, the \net{} will orient itself toward the
\zebra{}. This block was only made available in \stwo{}.

\begin{figure}[!t]
\centering
\includegraphics[width=3.3in]{graphs/approach_student_success.eps}
\caption{Shows the completion rate of each approach by student grouped by
  \sone{} and \stwo{}. An approach for a student is \com{} if any of the
  student's \com{} snapshots utilizes that approach. An approach for a student
  is \incom{} if they utilize the approach in any \incom{} snapshot and the
  approach is not found in any of the student's \com{} snapshots.}
\locallabel{fig:approach_student_success}
\end{figure}

\begin{figure}[!t]
\centering \includegraphics[width=3.3in]{graphs/approach_bar_Sequential2.eps}
\caption{Shows how many students utilized each approach in \stwo{} snapshots
  for three categories: \emph{all} snapshots, snapshots up to the first \com{}
  or last \incom{}, and the \emph{last} snapshot.}
\locallabel{fig:approach_bar_s2}
\end{figure}

A student may utilize a combination of these high-level approaches to complete
the assignment. For instance, in a single snapshot a student may use the
\emph{orient and glide} approach via a \rel{} orientation block to \catch{} the
\bear{} and subsequently use the \glideto{} approach to \catch{} the
\horse{}. Alternatively, students utilize several different classes of
orientation blocks. We wanted to see which combination of approaches was most
preferred amongst students who completed the assignment.

As mentioned in Section~\localref{sub:s1}, the base-project used as the
starting point for all students utilized an \abs{} orientation approach. In
order to accurately assess what approach combination the students explicitly
utilized, the code provided in the base-project was excluded from our approach
combination analysis.

Figure~\localref{fig:approach_student_success} shows a comparison of the
overall completion rate by student of each approach by assignment
iteration. Only snapshots up to a student's first \com{} snapshot are included
in this analysis as some teachers provided additional challenges to students
who had completed the assignment. The height of each bar indicates the total
number of students who had at least one snapshot that utilized the approach;
this value is provided as the upper-most number above the bar. The lower number
is the completion rate as a percentage, and the number within the lower segment
of the bar quantifies the number of \com{} students for the approach. An
approach is \com{} for a student if the student has at least one \com{}
snapshot utilizing that approach, otherwise an approach is \incom{} for a
student if none of their snapshots utilizing the approach are \com{}. An
approach is counted even when used in combination with another approach. Only
two and fifteen students utilized a combination of approaches in their first
\com{} snapshot for \sone{} and \stwo{} respectively.

This figure shows overwhelming evidence that students understood how to use
\glideto{} as the approach was \com{} for all but four students. Conceptually
this observation makes sense as the approach requires only a single block per
\catch{} rather than two or more blocks as required by other approaches.

The \abs{} approach had around a 50\% completion rate for both \sone{} and
\stwo{}. When considering that all students were provided with an \abs{}
approach example that \catch{es} the \zebra{}, this result indicates students
struggled with the \abs{} approach

While there are not many students for \sone{}, the figure does not convey that
all of the \com{} snapshots for the \rel{} approach belong to students in the
\emph{S1A} class. In fact, only one \emph{S1B} student attempted a \rel{}
approach, whereas all but four \emph{S1A} students attempted an \abs{}
approach. None of our field notes provide any insight as to why the \rel{}
approach was so prominent with the \emph{S1A} class, especially when compared
to the insignificance of the \rel{} approach with \stwo{}.

Finally for all students we look at the their usage of an approach across
snapshots in three categories:

\begin{description}
\item[all] the student utilized the approach in at least one snapshot,
  including snapshots made following a \com{} snapshot
\item[up to last] the student utilized the approach in at least one snapshot up
  to and including their first \com{} snapshot (includes all snapshots for
  students who had no \com{} snapshot)
\item[last] The approaches used only in either the first \com{}, or last
  snapshot for \incom{} students
\end{description}

Figure~\localref{fig:approach_bar_s2} quantifies the number of students who
utilize each approach in \stwo{} snapshots for each of the aforementioned
categories. There are two primary observations: The first is that for each
category the difference in height between the pink and yellow bars show the
number of students who abandoned an approach. The minuscule difference for the
\glideto{} approach provides additional evidence for the ease-of-use of that
approach. We make no claims about the abandonment of other approaches due to
the low number of students utilizing those approaches. The second observation
is the difference in height between the pink and green bars for each
category. This difference indicates student who tried a new approach only after
a \com{} snapshot. The figure shows very little growth in the \abs{} and \rel{}
approaches, but a nearly 15\% increase in the \glideto{} approach once again
providing support for the ease-of-use of the \glideto{} approach. A figure for
\sone{} is not provided as there are only two possible approaches, and none of
the students switched approaches after their first \com{} snapshot.

\subsection{Quantifying Students Affected by a Scratch Race Condition}
\locallabel{sub:race}

\begin{figure}[!t]
\centering \includegraphics[width=3.3in]{graphs/race_condition_Sequential1.eps}
\caption{Shows the breakdown of students affected by the race condition issue
  in Scratch for \sone{}.}
\locallabel{fig:rc_s1}
\end{figure}

\begin{figure}[!t]
\centering \includegraphics[width=3.3in]{graphs/race_condition_Sequential2.eps}
\caption{Shows the breakdown of students affected by the race condition issue
  in Scratch for \stwo{}.}
\locallabel{fig:rc_s2}
\end{figure}

Our curriculum development and test process as described in
Section~\localref{sec:classes} and visualized in Figure~\localref{fig:process}
allowed us to focus analysis on issues we were aware of based on the field
notes collected. However, the field notes were not always complete. During
manual assessment of the students' snapshots we noticed a number of snapshots
whose execution did not produce consistent results. These snapshots should have
consistently \caught{} the \zebra{}, however, only did so approximately 50\% of
the time. We discovered the problem to be a race condition within Scratch where
the detection of the intersection between two sprites may not occur in the
brief period of time that the sprites intersected. Instead, the next block in
the script, always a rotation block, would execute and the rotation would
result in separation of the two sprites; i.e., the two sprites were no longer
intersecting. Our in-class researchers confirmed having observed this issue,
however, we had no data to quantify the number of students affected. We
hypothesized that students affected by this issue may have struggled completing
the assignment.

In order to quantify the students affected we wrote a Hairball plugin to track
the \net{}'s \exe{}, that is, the sequence of blocks beginning with
\netclicked{}. We wanted to discover snapshots where the \net{}'s \exe{}
matches that of one we manually verified as exhibiting the race condition. We
labeled \exe{s} by programming them in Scratch, and executing the \sprogram{}
up to twenty times. If within these twenty executions we observed inconsistency
in the \catch{ing} of the \zebra{} the \exe{} was labeled as exhibiting the
race condition; otherwise it was not. While it is possible for a race condition
to emerge at a lessor frequency, we assume that in such cases few, if any,
students would experience the same low-frequency race condition. No race
condition exhibiting \exe{s} required more than eight executions to detect.

The result was a state machine that handled all \exe{s} of the \net{} shared by
more than any ten snapshots. We only handled \exe{s} up to the point that we
could label them as exhibiting the race condition or not. Of the 297 and 638
snapshots for \sone{} and \stwo{} respectively, only thirty-nine (13\%) and
thirty-eight (6\%) respective snapshots contained an \exe{} not explicitly
handled by our plugin.

In addition to labeling \exe{s} exhibiting the race condition, we labeled those
resulting in a consistent intersection with the \zebra{}. Students with such a
snapshot subsequent to a snapshot exhibiting the race condition are likely to
have expended effort to resolve the race condition. We label such cases as
\emph{fixed}. Additionally, for each student with one or more snapshots
exhibiting the race condition, we consider whether or not they completed the
assignment.

Figure~\localref{fig:rc_s1} and Figure~\localref{fig:rc_s2} show the breakdown
of all students who do or do not have at least one snapshot exhibiting the race
condition. In total eleven and forty-four students (38\% and 37\%) have
snapshots exhibiting the race condition for \sone{} and \stwo{}
respectively. Of those, three and nine students (27\% and 20\%) were unable to
complete the assignment. Six and twenty-four students (55\% and 55\%) took a
completely separate approach to completing the assignment after experiencing
the race condition, whereas only two and eleven students (18\% and 25\%) solved
the assignment by the addition of one or more blocks that explicitly prevented
the race condition.

Fortunately, a majority of students were able to avoid the race condition. We
sampled a few of those students and discovered the following three ways that
students completely avoided the race condition:

\begin{itemize}
\item The student immediately removed some or all of the provided code thus
  starting with a modified base-project.
\item In \stwo{}, the student simply appended \glideto{} blocks to the code.
\item The student immediately added an additional \glideDIST{} block resulting
  in consistent intersection between the \net{} and \zebra{}.
\end{itemize}

Of these three items, the middle item was how two of the thirteen students
resolved the race condition and solved the assignment, and the last item was
how the remaining eleven students did the same. In these two cases, the primary
difference between the labeling of students is that students who \emph{fixed}
the problem must have first had a snapshot exhibiting the problem.

The results show that over one third of students experienced the race
condition. Interestingly, the students who experienced the race condition were
statistically significantly more likely to complete the assignment: 73\% and
82\% compared to 39\% and 68\% (chi square, p < 0.028). This result was
unexpected, nevertheless, the labels provided by our static analysis allowed us
to discover the primary reason why students who did not experience the race
condition, did not complete the assignment. Manual inspection revealed that a
significant majority of these students' removed or significantly altered the
provided code in their first snapshot. The only other reason we discovered was
a handful of students used the last item from the list above to avoid the race
condition. In each of these cases it was apparent the addition was accidental
based on the subsequent modifications.

Overall, the effect of these results is that we were able to adjust our
assignment so that students are less likely to encounter a Scratch race
condition. Moreover, given the negative impact of students removing the
provided code, we learned that preventing students from modifying the provided
code may have a positive effect on learning.

\subsection{Snapshots Exhibiting the \emph{Double Click to Execute} Behavior}
\locallabel{sub:dce}
Scratch is built such that students can double click on any script, i.e., one
or more connected blocks, in order to execute that script. During \sone{},
in-class researchers noted that some students took advantage of this behavior
in order to execute scripts they created. While manually executing disjoint
scripts in this manner may trigger the success screen built into the
assignment, the researchers found that students exhibiting this behavior did
not understand the concept of a script. Instead these students viewed the
blocks as independent entities not triggered by an event (i.e., \netclicked)
and therefore these students did not exhibit the conceptual understanding we
had intended. Furthermore, staff members noted that students would \dce{} a
script in order to move from the start location to the first sprite, and then
alter that script to perform the next step of the sequence. Thus, as another
use of scaffolding in the assignment, we disabled the \dce{} feature in \stwo{}
in attempt to prevent students from going down an unintended path whilst
completing the assignment.

\subsubsection{Double Click to Execute Filters}
Once aware of the problem we sought to retroactively identify students who may
have utilized this \dce{} approach. We wrote a Hairball plugin to filter out
snapshots determined to not demonstrate the \dce{} behavior. The following
paragraphs detail, in-order, our approach to filtering snapshots to approximate
the students affected by the \dce{} behavior.

\textbf{\emph{Complete} Snapshots}: Any snapshots that when emulated by our
plugin result in the \net{} \catch{ing} any two or more sprites are
filtered. Furthermore, any chronologically subsequent snapshots by the same
student are filtered. The subsequent snapshots are filtered because, once a
student demonstrates success, we are not concerned about their \dce{} use.

\textbf{Motionless Snapshots}: Any snapshots that result in no movement due to
either having zero scripts or having only a single \netclicked{} script with no
movement blocks are filtered. These snapshots do not result in any motion and
thus are not indicative of the \dce{} behavior.

\textbf{\net{} Ends in Expected Location}: The Scratch save file stores the
location of all the sprites at the time the save occurred. We compare the
stored location of the \net{} to the final location of the \net{} as reported
by our \net{} emulation Hairball plugin. Snapshots containing movement whose
emulated \net{} location matches its stored location are filtered. These
snapshots are filtered because the \net{}'s stored location matches the
emulated location only when the in-Scratch execution matches the execution of
our emulation plugin and our emulation plugin does not support the \dce{}
behavior.

\textbf{Multiple \net{} Clicks}: We expect a student's script to execute only
once on \netclicked{} following a reset of the environment via a click on the
\emph{green flag}. However, it is possible to click the \net{} multiple times
resulting in an execution for each click where the \net{}'s position is not
reset. Thus, by performing the expected location test multiple times we are
able to both identify and filter snapshots exhibiting this multiple \net{}
click behavior. These snapshots are filtered for the same reason as those
filtered in the preceding paragraph.

\subsubsection{Double Click to Execute Snapshots}

\begin{figure}[!t]
\centering \includegraphics[width=3.3in]{graphs/dc_submissions.eps}
\caption{Depicts the number of \dce{} snapshots we identified for each
  student.}
\locallabel{fig:dc_snapshots}
\end{figure}

After applying all the filters, we counted the number of snapshots per student
that may exhibit the \dce{} behavior. Figure~\localref{fig:dc_snapshots} shows
the number of potential snapshots by student for both \sone{} and
\stwo{}. Recall that the \dce{} functionality was disabled completely in
\stwo{} thus we see that this filtering is not effective as it only removed
three of twenty-nine and thirty of 120 students respectively. We had hoped that
our filtering would result in a significant decrease in the number of snapshots
to manually inspect, however, that was not the case. Unfortunately without more
precise field notes, we cannot quantify the number of students affected as
there is not sufficient information in the snapshots for us to neither
automatically nor manually identify students exhibiting the \dce{} behavior.

It is important to note that while we could not proceed, it was not due to a
limitation of static analysis. The Hairball plugin we wrote considerably helped
us come to the conclusion that we simply had not gathered enough information to
neither manually nor automatically determine the students affected by the
\dce{} behavior. Based on this experience with a lack of data, we are altering
our data collection to capture every change made by students.
