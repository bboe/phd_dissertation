\section{Introduction}
Computer science is becoming one of the most ubiquitous areas of education at
the University level due to the importance of its core concepts within numerous
other fields of study, including materials science, biology, medicine, and
economics. The need for individuals to solve simple and complex problems using
computers continues to grow. While there is great demand in the work-force for
people with computer science experience, little effort has gone into classroom
curricula to prepare young students for careers involving computer science
skills. In fact, high school graduates have little, if any, idea of what
computer science entails.

The Computer Science Teachers Association (CSTA) has created a set of standards
in an effort to introduce students to computer science and to prepare them to
meet the demands of the future. These standards detail how to incorporate
computer science concepts into existing primary and secondary school
curricula~\cite{cstastandards}. The standards created by the CSTA are a
tremendous step in the right direction, shifting the focus of computer science
education from existing after-school outreach programs and summer camps to
required learning in schools~\cite{wayanoutpost, georgiaoutreach,
  Franklin:2011:ATA:1953163.1953295, Maloney:2010:SPL:1868358.1868363,
  Dann:2000:MCP:343048.343070, Hood:2005:TPL:1067445.1067454, csunplugged}.

The incorporation of computer science into \nth{4}--\nth{6} grade curricula
poses a challenge, as few research studies have focused on formal instruction
of computer science for this age group. We focus our attention toward this
effort. Using a design-based research approach, we begin with a simplified
version of our summer camp curriculum and deploy this curriculum in a number of
\nth{4}--\nth{6} grade classes in California~\cite{Franklin:2013:SBO}. Through
analysis of field notes collected by education researchers who observed many of
the students in these classrooms in combination with analysis of the students
in-progress work, we are able to identify computer science concepts that are
difficult for these students to understand, as well as identify other issues
with both the content of our curriculum, and our modified Scratch programming
environment. We use this knowledge to improve our curriculum, and then we
repeat this procedure with each wave of classes.

In this chapter, we focus on the use of static analysis to assist with computer
science curriculum development. We concentrate our analysis on a single Scratch
assignment that requires students to demonstrate the concept of sequential
execution by programming a \net{} to \emph{catch} three other sprites through a
sequence of actions.

The remainder of this chapter is organized as follows. We provide a brief
summary of related work in Section~\localref{sec:related}. In
Section~\localref{sec:methodology}, we present the methodologies used in our
study. We then present our results in Section~\localref{sec:results}. Finally,
Section~\localref{sec:conclusion} contains our conclusion.
