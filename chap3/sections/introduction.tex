\section{Introduction}
Computer science is quickly becoming one of the most ubiquitous areas of
education at the University level due to the application of its core concepts
in numerous other fields of study.  Moreover, the need for individuals to solve
both simple and complex problems using computers continues to grow. While there
is great demand in the workforce for people with computer science experience,
little effort in the classroom has gone into preparing young students for
careers involving some degree of computer science. In fact, despite the
prevalence of computer science, most high school graduates have little, if any,
idea of what computer science entails.

In an effort to introduce students to computer science and prepare them to meet
the demands of the future, the Computer Science Teachers Association (CSTA) has
created a set of standards.  These standards detail how to incorporate computer
science concepts throughout existing primary and secondary school
curricula~\cite{cstastandards}. The standards created by the CSTA are a
tremendous step in the right direction, shifting the focus of young student
computer science education from existing after-school outreach programs and
summer camps to required learning in schools~\cite{wayanoutpost,
  georgiaoutreach, Franklin:2011:ATA:1953163.1953295,
  Maloney:2010:SPL:1868358.1868363, Dann:2000:MCP:343048.343070,
  Hood:2005:TPL:1067445.1067454, csunplugged}.

The incorporation of computer science into \nth{4}--\nth{6} grade curricula
poses a challenge as little research has focused on formal instruction of
computer science for this age group. We focus our attention toward this
effort. Using a design-based research approach, we begin with a simplified
version of our summer camp curriculum and iteratively deploy this curriculum in
a number of \nth{4}--\nth{6} grade classes in
California~\cite{Franklin:2013:SBO}. Through a combination of an analysis of
field notes collected by education researchers who observed many of the
students in these classrooms, and an analysis of the students’ in-progress
work, i.e., snapshots, we are able to both identify computer science concepts
that are difficult for these students to understand, and identify other issues
present in our curriculum. We use this knowledge to iteratively improve our
curriculum and repeat this procedure with each wave of classes.

In this chapter, we focus on the use of static analysis to assist with
curriculum development. We concentrate our analysis on a single Scratch
assignment that requires students to demonstrate the concept of sequential
execution. In this assignment, students must program a \net{} to \emph{catch}
three other sprites by performing a sequence of actions.

The remainder of this chapter is organized as follows. We provide a brief
summary of related work in Section 3.2. In Section 3.3, we present the
methodologies used in our study. We then present our results in Section
3.4. Finally, Section 3.5 contains our conclusion.
