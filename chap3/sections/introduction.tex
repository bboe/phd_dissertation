\section{Introduction}
For decades, computer scientists have been developing activities for young
students in an effort to engage them in computer science early. These and other
efforts have led to many educational programming platforms, activities and
outreach programs~\cite{wayanoutpost, georgiaoutreach,
Franklin:2011:ATA:1953163.1953295, Maloney:2010:SPL:1868358.1868363,
Dann:2000:MCP:343048.343070, Hood:2005:TPL:1067445.1067454, csunplugged}. In
order to reach more students, computer science needs to be integrated into the
elementary school curriculum.

Towards this end, in 2011, the CSTA (Computer Science Teachers Association)
released a set of K-12 standards and scaffolded charts to illustrate
intermediate points~\cite{cstastandards}. While an important step, much work
remains before computer science can develop research-based curricula. At a
high level, two large gaps remain: empirical evidence to confirm and refine the
information presented in the scaffolding charts, and lower anchor points to
determine appropriate starting points for the curriculum.

In order to gain deeper insight into how young students (4th-6th grade
students, ages 9-12) learn computer science, we modified an existing
middle-school curriculum to be appropriate for 4th grade. This curriculum
introduced the concepts necessary for digital storytelling in Scratch:
sequential execution, event-based programming, initialization, message-passing,
costume changes, and scene changes~\cite{Franklin:2013:SBO}.

We used a design-based research study approach to creating and modifying our
curriculum. We worked with five schools with staggered start dates. For local
schools, our education researchers took field notes about challenges students
faced in the classroom, and our computer scientists inspected student projects
for more detailed information. After discussion by all researchers, partial
understandings and barriers to full understanding were identified. These were
used to adjust the curriculum and programming environment for the current and
future schools.

In this study we focus on the use of static analysis to assist with curriculum
development. We concentrate our analysis on a single Scratch assignment that
requires students to demonstrate the concept of sequential execution. In this
assignment students must program a \net{} to \emph{catch} three other sprites
by performing a sequence of actions.

The remainder of the study is organized as follows. We provide a brief summary
of related work in section~\localref{sec:related}. In
section~\localref{sec:methodology} we present the methodologies derived from
educational research used in our study. We then present our results in
section~\localref{sec:results}. Finally, section~\localref{sec:conclusion}
contains our conclusion.
