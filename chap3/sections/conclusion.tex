\section{Conclusion} \locallabel{sec:conclusion}
This study detailed our continued use of Hairball for assessment of
Scratch-based assignments. We described our modifications to Scratch in order
to apply instructional scaffolding, and presented the results of two iterations
of our sequential execution assignment. Two goals of our assignment were that
students would recognize the need to add additional blocks to the base-project,
and understand the importance of block ordering in order to demonstrate
proficiency of sequential execution in Scratch. In this section we state our
conclusions regarding improvements made to our curriculum, and to the use of
static analysis as a curriculum development tool.

\subsection{Curriculum Improvements}
We created and utilized Hairball plugins to help answer the three questions we
sought to answer (Section~\localref{sec:results}). In total, 102 of the 149
students for whom we had consent completed the assignment. While an overall
68\% is not impressive, the percentage increased from 52\% to 73\% due to
improvements we made after \sone{}. While there is more room for improvement
with respect to the success of the students, we consider this increase to be a
success of the improvements we made to both our Scratch interface and the
curriculum.

The single most important change we made was the addition of the \glideto{}
block as it enabled the use of only a single block for each of the three
essential \emph{pick up} actions. Recall that our goal was not for students to
understand position and orientation changes, but simply for them to program
sequential code that \emph{picks up} all the objects. Based on these results,
we believe additional instructional scaffolding would be beneficial. Students
should first be asked to solve the challenge using only \glideto{}, and once
mastering that task, should be challenged with a similar task using only one of
the \emph{Orient and Glide} approaches. In both cases, only the necessary
blocks should be available for students to use.

\subsection{Static Analysis}
Hairball plugins were written to quantify students challenged with issues
identified by education researchers in the field notes. We described one such
plugin identifying that 40\% of all students experienced a Scratch race
condition. Interestingly, 78\% of those students completed the assignment
indicating a statistically significant correlation between experiencing the
race condition and completing the assignment. While we were successful in
writing automated assessment for the race condition issue, we could not do the
same for the \dce{} issue. The plugin was able to filter half the snapshots,
however, the root issue was that even with manual analysis we could not
precisely differentiate between snapshots exhibiting this behavior and normal
behavior due to the lack of information contained in the snapshots. In this
regard, we consider the use of the plugin a success as it helped us swiftly
determine the large subset of submissions that may exhibit the behavior. This
information, in turn, permitted us to come to the aforementioned conclusion.

The two most significant benefits of using static analysis in assignment
assessment are the speed of assessment, and the accuracy of assessment. While
there is overhead involved in creating static analysis, it is a one-time
overhead with essentially infinite scaling capabilities. The overhead for
training a human, on the other hand, may require less time, but has no scaling
capabilities. Furthermore, static analysis will consistently produce the same
results, whereas humans are significantly less likely to do so.

Another significant advantage of incorporating static analysis in assignment
assessment is due to the dramatic reduction in overhead required with each
iteration of assessment criteria; of which, we had many. With only the addition
of a short amount of time required to adapt our static analysis to updated
assessment criteria, we were otherwise able to rerun the entire modified
assessment across all snapshots in the matter of minutes. A human, conversely,
could at best assess six snapshots in a minute. Assuming that is feasible, each
assessment criteria iteration would have required 2.6 hours for analysis of our
935 snapshots. While, in general, the number of assessment criteria iterations
can be reduced with more in-depth up-front preparation, using static analysis
permits a flexibility in assignment assessment that is not limited by human
factors.

Finally, the plugins written for our assessment of this assignment will be used
in future iterations of the assignment to support additional interface and
curriculum changes. The use of Hairball plugins in our assessment shows the
usefulness of static analysis tools in the development of \nth{4}--\nth{6}
grade curriculum. In the future, we hope to incorporate the use of these
plugins to provide real-time feedback and assessment to students and
instructors as students work on the assignment through the use of an automated
snapshot collection system.
