\chapter{Using Static Analysis to Assist with the Development of a
  Scratch-based 4th--6th Grade Classroom Curriculum}
\label{chap:curriculum}

\def\currentprefix{curriculum}

In the previous chapter, we detailed the role that static analysis plays in the
post-assessment of Scratch-based \nth{4}--\nth{6} grade summer camp
assignments. We showed that static analysis assisted assessment both by
reducing the speed of assessment, and by improving the accuracy of
assessment. In this chapter, we expand on the use of static analysis for the
assessment of \nth{4}--\nth{6} grade Scratch assignments by extending its use
to the context of classroom curriculum development. We wanted to increase the
accuracy of our assessment of student learning by improving the interface and
assignments that make up the curriculum. Here we explore the possibility of
using feedback from static analysis of student-created \sprogram{s} to direct
changes to both the Scratch interface and our curriculum. We found that static
analysis supports rapid iteration of this curriculum development cycle.

\subimport{sections/}{introduction}
\subimport{sections/}{relatedwork}
\subimport{sections/}{methodology}
\subimport{sections/}{results}
\subimport{sections/}{conclusion}
