\chapter{Introduction} \label{chap:introduction}
A recent study by code.org suggests that by 2020 there will be a one million
person gap in the United States between the number of vacant computer science
jobs and the number of computer scientists available to fill these
jobs~\cite{codeorg:2013}. The lack of newly trained computer scientists is a
twofold problem:

\begin{itemize}
\item First, there are not enough students interested in computer science
  coming out of high school, evidenced by the fact that of the 2.1 million
  students nationwide who took AP exams in 2013, only 31,000 (1.4\%) students
  took the AP Computer Science exam~\cite{cb:2013}. The low numbers of high
  school students studying computer science is a reflection of the absence of
  computer science-related instruction in areas such as computational thinking
  and elementary programming in primary and middle schools.
\item Second, at this point in time, the university system does not have the
  resources to adequately handle the rapid, yet insufficient, increase in
  numbers of students applying to computer science programs. The future does
  not look much more encouraging, as departments may not be able to support
  computer science enrollment growth through an increase in resources including
  faculty size. For example, Lazowska et al.\ report that while the student
  body of both Princeton and MIT comprises more than 10\% computer science
  majors, it is unlikely that 10\% of the total university faculty will ever be
  part of computer science~\cite{lazowska:2014}. The unfortunate result is that
  university computer science departments are turning away a significant number
  of qualified students from a discipline where they are severely needed.
\end{itemize}

This dissertation describes methods I developed along with my colleagues that
have a positive impact on solving both the challenge to get more students
interested in computer science coming out of high school, as well as the
challenge to support the growth in number of university-level computer science
students. My work helps increase student interest in computer science through
the introduction of a \nth{4}--\nth{6} grade Scratch-based computational
thinking curriculum. My work also supports growth in the number of
university-level computer science students by incorporating a real-time
feedback and assessment system into existing computer science curriculum. The
system I developed reduces the amount of time that instructors currently devote
to the labor-intensive assessment process, thereby making more time available
to spend with students who need extra assistance. My research, in combination
with future efforts, will enable the university system to produce more computer
scientists.

\section{Thesis Statement}
The increase in interest in computer science has resulted in a need to scale
computer science instruction from the primary school grades through
undergraduate level university programs. These two ends of the spectrum,
however, are in far different places in their development, with very little
curriculum existing for primary schools and very mature instruction available
at the university level. Assessment automation can greatly enhance both efforts
by allowing us to understand certain important aspects of student learning
behavior. At the primary school level, assessment automation through static
analysis of student work can provide instructors with insight into student
comprehension, enabling rapid curriculum changes that result in faster
deployment of new curriculum. At the university level, automated feedback and
assessment systems provide large numbers of students with immediate insight
into their success with class assignments allowing them to iteratively achieve
mastery of course topics, and reduce assessment time permitting instructors to
focus their efforts on students in need.

\section{Dissertation Overview}
The remainder of this dissertation is structured as follows. In
Chapter~\ref{chap:hairball} I look at the use of static analysis to assist with
the \emph{post assessment} of five Scratch assignments given in a two-week
Scratch-based \emph{summer camp} for \nth{6}--\nth{8} grade students. In
Chapter~\ref{chap:curriculum} we extend the use of static analysis of Scratch
assignments to aid in the \emph{development} of a \nth{4}--\nth{6} grade
\emph{classroom-based computational thinking
  curriculum}. Chapter~\ref{chap:feedback} looks at \emph{submission behaviors}
of university-level computer science students in the presence of our real-time
feedback and assessment system. Finally, in Chapter~\ref{chap:conclusion} we
summarize our findings and discuss the impact of our research on computer
science education at both the primary and university instructional levels.
