\chapter{Introduction} \label{chap:introduction}
A recent study by code.org suggests that by 2020 there will be a one million
person gap in the United States between the number of vacant computer science
jobs and the number of computer scientists available to fill these
jobs~\cite{codeorg:2013}. Despite an increase in enrollment in computer science
undergraduate programs since 2008, it is projected that the university system
will not produce enough graduates with computer science or related degrees to
meet this demand.

This lack of newly trained computer scientists is a twofold problem. First,
there are not enough students interested in computer science coming out of high
school. This is evidenced by the fact that of the 2.1 million students
nationwide who took an AP exam in 2013, only 31,000 (1.4\%) students took the
AP Computer Science exam~\cite{cb:2013}. This problem is a reflection of the
absence of computer science related instruction in areas such as computational
thinking and elementary programming in primary and middle schools.

Second, at this point in time, the university system does not have the
resources to adequately handle the number of students applying to computer
science programs, and even in the future may not be able to support computer
science enrollment growth through an increase in faculty size. For example, in
a recent presentation, Lazowska et al. report that while the student body of
both Princeton and MIT comprises more than 10\% computer science majors, it is
unlikely that 10\% of the total university faculty will ever be part of
computer science~\cite{lazowska:2014}. Thus, this increase in demand, along
with the inability to keep up with the demand, results in university computer
science departments turning away a significant number of people from an area
where they are severely needed.

This dissertation describes methods I developed along with my colleagues that
have a positive impact on solving both the challenge to get more students
interested in computer science coming out of high school, as well as the
challenge to support the growth in the number of university-level computer
science students. My work helps increase student interest in computer science
through the introduction of a \nth{4}--\nth{6} grade Scratch-based
computational thinking curriculum. My work also supports growth in the number
of university-level computer science students by introducing a real-time
feedback and assessment system into existing computer science curriculum.  I
feel that my work, in combination with future efforts, will help close the gap
between the number of newly trained computer scientists needed in the work
force, and those produced by the university system.

\section{Thesis Statement}
The increase in interest in computer science has resulted in large demand for
computer science instruction, from primary school grades through undergraduate
level university programs. These two ends of the spectrum, however, are in far
different places in their development, with very little curriculum existing for
primary schools and very mature instruction available for
universities. Assessment automation can greatly enhance both efforts. At the
primary school level, assessment automation through static analysis of student
work can provide instructors with insight into student comprehension, resulting
in rapid curriculum changes. This insight, however, is limited by the structure
of assignments and the amount of data collected. At the university level,
automated assessment can provide students with insight into their success on
class assignments allowing them to iteratively achieve mastery. However, the
timing and quantity of feedback may inhibit students' from developing important
secondary processes such as testing and debugging.

\section{Dissertation Overview}
The remainder of this dissertation is structured as follows. In
Chapter~\ref{chap:hairball} I look at the use of static analysis to assist with
the \emph{post assessment} of five Scratch assignments given in a two-week
Scratch-based \emph{summer camp} for \nth{4}--\nth{6} grade students. In
Chapter~\ref{chap:curriculum} we extend the use of static analysis of Scratch
assignments to aid in the \emph{development} of \nth{4}--\nth{6} grade
\emph{classroom-based computational thinking
  curriculum}. Chapter~\ref{chap:feedback} looks at \emph{submission behaviors}
of university-level computer science students in the presence of our real-time
feedback and assessment system. Finally, in Chapter~\ref{chap:conclusion} we
summarize our findings and discuss the impact of our research on computer
science education at both the primary and university instructional levels.
