\chapter{Conclusion} \label{chap:conclusion}

As discussed in this dissertation, it is critical that we in the computer
science community do whatever possible to increase the number of new computer
scientists and prepare them for the challenges they must meet in industry,
education, medicine, science, and so on. The increase in enrollments in
response to job demand will impact all levels of the educational system. My
research contributes to this effort. First, I demonstrate the effectiveness of
static analysis in both the post-assessment of a Scratch-based \nth{6}--\nth{8}
grade summer camp and the development of a Scratch-based \nth{4}--\nth{6} grade
classroom curriculum. Second, I report on the submission behavior of university
computer science students in the presence of a real-time feedback and
assessment system. The significance of this collective research is to support
the growth in number of students who seek computer science education, and to do
so while maximizing student performance.

My work is but a single step on the journey to increasing the yearly number of
new computer scientists by both increasing student interest in computer science
and maximizing the learning potential of those studying computer
science. Continued research across all levels of computer science curricula
from primary school through university is necessary to complete this
journey. In primary school, static analysis can be incorporated into the
student feedback and assessment cycle much like I have done with university
assignments. However, it is unknown how younger students will respond to such
feedback, thus there is much to be done with respect to how best to provide
feedback for students of various ages and topic mastery. In both areas, the
continued application of machine learning across collected data sets can be
used to understand how students best solve certain programming assignments in
order to differentiate successful approaches from unsuccessful approaches to
solving common programming problems.

I envision a future where years from now, many students with exposure to and
successful completion of our primary school computer science related curriculum
will ultimately choose a college major that involves some degree of
computational thinking. Their university assignments will include electronic
submission, and will be designed by their instructors to provide them with
optimal feedback at the optimal time to maximize their understanding in the
least amount of time. Students of the future will spend less time to learn more
and instructors will have more time to work with students requiring additional
assistance.

This evolution in student learning is possible only through analysis and
assessment of student in-progress work via studies similar to those I
performed. The iterative application and subsequent measurement of new and
altered techniques will not only advance computer science education at a
fundamental level, it will also make it possible to educate increased numbers
of computer science students without a proportional increase in instructional
resources. The resulting cohort of well-educated computational thinkers with
shared knowledge and concrete skill sets will be able to solve more of our
real-world problems. I hope we will all see this future.
