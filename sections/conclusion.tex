\chapter{Conclusion} \label{chap:conclusion}

In this dissertation I have demonstrated the effectiveness of static analysis
both in the post-assessment of a Scratch-based summer camp, and in the
development of a \nth{4} -- \nth{6} grade Scratch curriculum. And I reported on
the submission behavior of college students in the presence of a real-time
feedback and submission system. The significance of this collective research is
to support both the growth in number of students who seek computer science
education, and to do so while maximizing student performance.

I envision a future, where years from now, many students who will have
completed our primary school curriculum will face a decision to choose a
college major. Many of these students will select a major involving some degree
of computation thinking. Their assignments will be electronically submitted,
and specifically designed to provide them the optimal amount of feedback at the
optimal time to maximize their understanding in the minimum amount of time
enabling students of the future to learn a significant amount more in a shorter
amount of time.

These potential advancements in student learning are only possible due to
analysis and assessment of student in-progress work that occur via studies
similar to mine. Every bit of information expands our general knowledge of
student learning. Only through the iterative application and subsequent
measurement of a new or altered technique can computer science education
advance to the point where this future becomes reality. I hope we all see this
future.
