\chapter{Conclusion} \label{chap:conclusion}

As discussed in Chapter~\ref{chap:introduction}, it is critical that the
computer science community do whatever we can to increase the number of new
computer scientists and prepare them for the challenges they must meet in
industry, education, medicine, science, and so on. The increase in enrollments
in response to the job demand will impact all levels of the educational
system. In this dissertation I have contributed to this effort. First, I
demonstrated the effectiveness of static analysis in both the post-assessment
of a Scratch-based \nth{4}--\nth{6} grade summer camp, and the development of a
Scratch-based \nth{4}--\nth{6} grade classroom curriculum. Second, I reported
on the submission behavior of university computer science students in the
presence of a real-time feedback and assessment system. The significance of
this collective research is to support the growth in number of students who
seek computer science education, and to do so while maximizing student
performance.

I envision a future, where years from now, students who have completed our
primary school curriculum will face a decision to choose a college major. Many
of these students will select a major involving some degree of computational
thinking, including computer science. Their assignments will be electronically
submitted, and designed specifically to provide them with optimal feedback at
the optimal time to maximize their understanding in the least amount of
time. Thus, enabling students of the future to spend less time to learn more.

These advancements in student learning are possible only through analysis and
assessment of student in-progress work via studies similar to mine. Every bit
of information expands our general knowledge of student learning. Only through
the iterative application and subsequent measurement of new or altered
techniques can computer science education advance to the point where this
future becomes reality. In this reality of dramatically increased numbers of
well-educated computational thinkers, their shared knowledge and concrete skill
sets will make it possible for them to solve more of our real-world problems. I
hope we all see this future.
