\chapter{Conclusion} \label{chap:conclusion}

As discussed in Chapter~\ref{chap:introduction}, it is critical that we in the
computer science community do whatever possible to increase the number of new
computer scientists and prepare them for the challenges they must meet in
industry, education, medicine, science, and so on. The increase in enrollments
in response to job demand will impact all levels of the educational system. My
research contributes to this effort. First, I demonstrate the effectiveness of
static analysis in both the post-assessment of a Scratch-based \nth{4}--\nth{6}
grade summer camp and the development of a Scratch-based \nth{4}--\nth{6} grade
classroom curriculum. Second, I report on the submission behavior of university
computer science students in the presence of a real-time feedback and
assessment system. The significance of this collective research is to support
the growth in number of students who seek computer science education, and to do
so while maximizing student performance.

I envision a future where years from now, many students who complete our
primary school computer science related curriculum will ultimately choose a
college major that involves some degree of computational thinking. Their
university assignments will be electronically submitted, and designed
specifically to provide them with optimal feedback at the optimal time to
maximize their understanding in the least amount of time. Students of the
future will spend less time to learn more.

This evolution in student learning is possible only through analysis and
assessment of student in-progress work via studies similar to mine. The
iterative application and subsequent measurement of new and altered techniques
will not only advance computer science education at a fundamental level, it
will also make it possible to educate increased numbers of computer science
students without an increase in instructional resources. The resulting cohort
of well-educated computational thinkers with shared knowledge and concrete
skill sets will be able to solve more of our real-world problems. I hope we all
see this future.
