\begin{abstract}

\addcontentsline{toc}{chapter}{Abstract}

There is an increasing demand for newly trained computer scientists as the
number of computer science related jobs continues to increase. While computer
science undergraduate programs have seen a growth in enrollment since 2008, the
rate of growth is not sufficient stemming from a lack of interest in young
students. Despite the growing increase in demand for computer science
professionals, computer science at the primary school level is virtually
nonexistent. As a result, significant effort is being made to incorporate
computational thinking into existing primary school education in order to
increase young students' interest in computer science. However, validating and
deploying new curriculum across the nation is a great challenge.

To enable wide-scale computer science education we do two things. First, we
create a framework called Hairball to support the static analysis of
\sprogram{s}. Scratch is a popular building-block language utilized to pique
interest in, and teach the basics of computer science. We describe how we
utilize Hairball to verify and test that learning objectives are met in our
pilot computational thinking curriculum targeted for fourth, fifth and sixth
grade students. Hairball allows for rapid curriculum alterations and thus
contributes to wide-scale deployment of \nth{4}--\nth{6} grade computer science
curriculum.

Second, we create a real-time feedback and assessment system utilized
extensively by students of two UCSB courses since winter quarter 2013. The
configuration of the system was systematically varied from assignment to
assignment and between instances of the courses to measure changes in student
submission behavior. We show that changes to the system configuration affect
student submission behavior. For example, increasing the time between when
students make a submission and when they receive feedback correlates with an
increase in score improvement between submissions. This system, along with its
findings, has a positive impact on supporting wide-scale computer science
education by allowing instructors to tailor assignments, and associated
assignment parameters to optimize student learning.

\end{abstract}
