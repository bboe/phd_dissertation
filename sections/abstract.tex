\begin{abstract}

\addcontentsline{toc}{chapter}{Abstract}

Enrollment in computer science undergraduate programs has increased nationally
since 2008. Meanwhile the demand for college graduates to fill computer science
related jobs continues to increase. In order to produce more computer science
graduates, computer science departments must increase their enrollment
sizes. In lieu of hiring additional faculty, many departments are investigating
methods for scaling the number of students. Despite the growing increase in
demand for computer science professionals, computer science at the primary
school level is virtually nonexistent. As a result, significant effort is being
made to incorporate computational thinking into existing primary school
education. Validating and deploying new curriculum across the nation is a great
challenge.

To enable wide-scale computer science education we do two things. First, we
created a framework to support the static analysis of \sprogram{s}, named
Hairball. Scratch is a popular building-block language utilized to pique
interest in, and teach the basics of computer science. We describe how we
utilize Hairball to verify and test that our learning objectives are met in our
pilot computational thinking curriculum targeted for fourth, fifth and sixth
graders. The use of Hairball allows for rapid curriculum alterations and thus
contributes to wide-scale deployment of \nth{4} -- \nth{6} grade computer
science curriculum.

Second, we created a real-time feedback and assessment system that has been
utilized extensively by students of two UCSB courses since winter quarter
2013. The configuration of the system was systematically varied from assignment
to assignment and between instances of the courses to measure changes in
student submission behavior. We show that changes in the system configuration
affect student submission behavior. This system, along with its findings, has a
positive impact on supporting wide-scale computer science education.

\end{abstract}
