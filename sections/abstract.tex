\begin{abstract}

\addcontentsline{toc}{chapter}{Abstract}

There is an increasing demand for newly trained computer scientists as the
number of computer science related jobs continues to increase. While computer
science undergraduate programs have seen a growth in enrollment since 2008, the
rate of growth is not sufficient to meet this demand, stemming from a lack of
interest from young students in the primary and secondary educational
levels. As a result, significant effort is being made to incorporate
computational thinking into existing primary school education in order to
increase young students' interest in computer science. However, validating and
deploying new curriculum across the nation is a tremendous challenge for
computer science education researchers and curriculum designers.

To enable wide-scale computer science education we do two things. First, we
create a framework called Hairball to support the static analysis of
\sprogram{s}. Scratch is a popular building-block language utilized to pique
interest in, and teach the basics of computer science. We describe how we
utilize Hairball to ensure that learning objectives are met in our pilot
computational thinking curriculum targeted for fourth, fifth, and sixth grade
students. Hairball allows for rapid curriculum alterations and thus contributes
to wide-scale deployment of \nth{4}--\nth{6} grade computer science curriculum.

Second, we create a real-time feedback and assessment system utilized
extensively by students in a total of seven instances of two UCSB computer
science courses from Winter quarter 2013 through Spring 2014. The configuration
of the system was systematically varied from assignment to assignment and
between instances of each course to measure changes in student submission
behavior. We show that changes to the system configuration affect student
submission behavior. For example, increasing the time between when students
make a submission and when they receive feedback correlates with an increase in
score improvement between submissions. The insights from our analysis of
student submission behavior have a positive impact on supporting wide-scale
computer science education by allowing instructors to tailor assignments and
associated assignment parameters to optimize student learning.

\end{abstract}
