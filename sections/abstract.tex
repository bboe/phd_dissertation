\begin{abstract}

\addcontentsline{toc}{chapter}{Abstract}

There is a proliferating demand for newly trained computer scientists as the
number of computer science related jobs continues to increase. University
programs will only be able to train enough new computer scientists to meet this
demand when two things happen: when there are more primary and secondary school
students interested in computer science, and when university departments have
the resources to handle the resulting increase in enrollment. To meet these
goals, significant effort is being made to both incorporate computational
thinking into existing primary school education, and to support larger
university computer science class sizes.

To enable wide-scale computer science education we do two things. First, we
create a framework called Hairball to support the static analysis of
\sprogram{s} targeted for fourth, fifth, and sixth grade students. Scratch is a
popular building-block language utilized to pique interest in and teach the
basics of computer science. We observe that Hairball allows for rapid
curriculum alterations and thus contributes to wide-scale deployment of
computer science curriculum. Second, we create a real-time feedback and
assessment system utilized in university computer science classes to provide
better feedback to students while reducing assessment time. Insights from our
analysis of student submission data show that modifications to the system
configuration support the way students learn and progress through course
material, making it possible for instructors to tailor assignments to optimize
learning in growing computer science classes.

\end{abstract}
