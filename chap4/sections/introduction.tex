\section{Introduction}
The growing demand for Computer Science education is resulting in a shift
toward larger class sizes. In order to accommodate larger class sizes, many
Computer Science classes have begun to utilize automated assessment technology
where students submit their work electronically, and a significant portion of
the assessment is performed by pre-written test cases and static
analysis. Furthermore, a subset of these automated assessment systems provide
students with real-time feedback and unlimited submission attempts up to the
deadline thus permitting students to iteratively achieve mastery on their
assignments. While these submission systems support scaling class sizes without
much increase in human resources, little is known about the impact of such
systems on student learning.

We created and deployed a real-time feedback and submission system for the
purposes of supporting scale in UCSB Computer Science lower division
courses. Anecdotal evidence suggested that while the system permitted students
to achieve success on assignments, the students were observed to rely on the
system rather than develop their own testing and debugging skills; skills they
were forced to develop in order to succeed in the absence of feedback.

While our curriculum is not designed to assess students testing and debugging
skills, we sought to measure the effect of delaying feedback students receive
in hope of discouraging reliance upon the real-time feedback and submission
system. In this study we present the results obtained by an analysis of 20,777
submissions collected from six classes between Winter quarter 2013 and Spring
quarter 2014. We provide a general overview of student submission behavior in
the presence of a real-time feedback and submission system, and provide an
analysis of the feedback delay's effect on student submission behavior.

The remainder of this study is organized as follows. We provide a brief summary
of related work in Section~\localref{sec:relatedwork}. In
Section~\localref{sec:methodology} we describe the methodology of our study. We
then present our results in Section~\localref{sec:results}, and finally,
conclude in Section~\localref{sec:conclusion}.
