\section{Conclusion} \locallabel{sec:conclusion}
In this study we discover student submission behaviors through the analysis of
20,777 submissions made by 289 students across seven classes. The data was
collected by a real-time feedback and assessment system we created that allowed
a per-assignment feedback delay to be configured. Our results show that
delaying feedback impacts student submission behavior. Furthermore, delays of
at least thirty minutes positively affect submission efficiency, as we defined
it, when compared to smaller delays. Our results also suggest that delaying
feedback impacts student work sessions in two ways:

\begin{itemize}
\item Increases in delay correlate with longer work sessions
  (Section~\localref{sec:efficiency}).
\item Increases in delay correlate with less improvement during work sessions
  (Section~\localref{sec:session_impact}).
\end{itemize}

The aggregate result of the feedback delay suggests that assignments should be
configured with a thirty minute feedback delay. Our results also provide an
interesting comparison to prior work:

\begin{itemize}
\item We confirm that starting early correlates with higher assignment
  scores (Section~\localref{sec:early}).
\item We confirm a high period (not peak) of student activity between \PM{4}
  and \PM{6} regardless of deadline (Section~\localref{sec:time}).
\item We identify peak student activity occurring in the hours prior to the
  time of a deadline both on the day of the deadline and others. Due to our
  differences with prior work, we hypothesize that students adjust their peak
  working hours to align with pending deadlines (Section~\localref{sec:time}).
\item We confirm that a majority of activity is completed in the two days prior
  to assignment deadline (Section~\localref{sec:efficiency}).
\end{itemize}

In addition to our comparison with prior work we offer some other new insights:

\begin{itemize}
\item We discover that there is no difference in submission efficiency due to
  proximity with an assignment deadline (Section~\localref{sec:efficiency}).
\item We show that students take advantage of the ability to continue using the
  feedback and assessment system in order to receive feedback after the
  deadline (Section~\localref{sec:deadline}).
\item We show that, within reason, the selection of the window size used to
  group submissions into work sessions is irrelevant
  (Section~\localref{sec:window}).
\end{itemize}

Overall, this study provided an insight into a few aspects of student
submission behavior. While more research is required to fully understand
student submission behavior, our results should help guide instructors toward
ideal assignment configuration with respect to feedback delay and assignment
deadlines in effort to improve student success.
