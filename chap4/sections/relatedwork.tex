\section{Related Work}\locallabel{sec:relatedwork}

A number of educators have designed and built automated assessment and feedback
systems going back in time as far as 1960. A survey of the history and
application of these systems was performed by Douce et al.\ in
1995~\cite{Douce:2005:ATA:1163405.1163409}. Ihantola et al.\ picks up where
Douce et al.\ left off with a review of automated assessment and feedback
system literature published between 2006 and
2010~\cite{Ihantola:2010:RRS:1930464.1930480}. Despite the plethora of
publications pertaining to these systems, little was reported on student
submission behavior in the presence of these systems. Only recently have
researchers begun to look at the behavior of students utilizing automated
feedback and assessment systems in order to gain insight into behaviors that
are more likely to contribute to successful assignment completion.

\spacco{} analyzed over 37,000 snapshots from ninty-six students collected
using their Marmoset automated assessment and feedback system in Spring
2006. They correlated both starting early with better final scores, and the
length of a work session with score improvement. In attempt to encourage
students to start assignments earlier than they would normally, Marmoset was
designed with a renewable token component that would permit students to receive
feedback from additional assignment test cases at most three times in a one-day
period. \spacco{} report, however, that their data do not show significant
evidence of students starting earlier in order to be able to utilize additional
tokens~\cite{Spacco:2013:TIP:2462476.2465594, Spacco:2006:EMD:1140124.1140131}.

Edwards et al.\ analyzed nearly 90,000 assignment submissions from 1,101
students collected over a five year period beginning in Spring 2004 using
Edwards's Web-CAT automated assessment and feedback
system~\cite{Edwards:2003:RCS:949344.949390}. Edwards et al.\ found that, among
students who did not score consistently across assignments, these students both
started and finished earlier when receiving an \emph{A} or \emph{B} score, than
when receiving a \emph{C}, \emph{D}, or \emph{F} score. Furthermore, they also
showed a correlation between starting earlier and assignment
score~\cite{Edwards:2009:CEI:1584322.1584325}.

Helminen et al.\ reported on student programming and testing behaviors
collected by their online code editor and execution environment in Fall
2012. While students were only required to submit their assignments through the
environment, many used it for development and testing. With this environment,
Helminen et al.\ were able to capture detailed student activity including when
students started and stopped working, edits made to their code and associated
tests, commands issued for testing, and when students made a submission.  They
found that few students took advantage of the automated feedback provided by
the system. Helminen et al.\ speculated this result was due to the already
significant test coverage from test cases provided with the
assignments~\cite{Helminen:2013:RAI:2526968.2526970}.

Most recently, Falkner et al.\ looked at the impact of the granularity of
assignment scores on student submission behavior. They found that student
scores improved with an increase in assignment score
granularity~\cite{Falkner:2014:IEA:2538862.2538896}. While it may seem
intuitive that assignments with more precise scoring will generally improve
student scores, their research provides evidence supporting that claim.

Overall, the impact of these studies are an ever increasing corpus of student
submission behavior contained within an extraordinary number of assignment
submissions. These submissions and the student behavior they represent contain
a surfeit of knowledge that computer science education researchers have only
just begun to understand. In this study, we hope to reveal some of this
knowledge by comparing our analysis results with previous results, and by
looking at the impact of a delay in feedback on student submission behavior.
