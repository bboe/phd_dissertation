\section{Related Work}\locallabel{sec:background}

A number of educators have designed and built automated assessment systems. Two
notable surveys of the this field were completed by Douce et al., and Ihantola
et al.~\cite{Douce:2005:ATA:1163405.1163409,
  Ihantola:2010:RRS:1930464.1930480}.

Recently, a number of studies have looked at the behaviors of students
utilizing automated feedback and assessment systems in order to gain insight on
behaviors more likely to contribute to successfully completing assignments.

\spacco{} analyzed over 37,000 snapshots from 96 students collected using their
Marmoset automated grading system. They correlated both starting early with
better final scores, and the length of a work session with score
improvement. Marmoset's token system was designed to encourage students to
start earlier, however, the data did not show significant evidence of students
starting earlier in order to take advantage of additional release
tokens~\cite{Spacco:2013:TIP:2462476.2465594, Spacco:2006:EMD:1140124.1140131}

Similarly, Edwards et al. analyzed nearly 90,000 assignment submissions from
1,101 students collected by Edwards's Web-CAT automated grading
system~\cite{Edwards:2003:RCS:949344.949390}. Edwards et al. found that, among
students who did not consistently have similar scores, these students both
started and finished earlier when receiving an \emph{A} or \emph{B} score, than
when receiving a \emph{C}, \emph{D}, or \emph{F} score. Furthermore, they also
showed a correlation between starting earlier and assignment
score~\cite{Edwards:2009:CEI:1584322.1584325}.

Helminen et al. reported on student programming and testing behaviors collected
by an online code editor and execution environment they created. While students
were only required to submit their assignments through the environment, many
used it for development and testing. With this environment Helminen et al. were
able to capture detailed student activity including when they started and
stopped working, edits made to their programs and associated tests, commands
issued for testing, and when students made a submission.  They found few
students took advantage of their automated feedback, likely due to the
significant test coverage from test cases provided with the
assignments~\cite{Helminen:2013:RAI:2526968.2526970}.

Most recently, Falkner et al. looked at the impact of the granularity of
assignment scores on student submission behavior. They found that student
scores improved with an increase in assignment score
granularity~\cite{Falkner:2014:IEA:2538862.2538896}.

The impact of these studies are a growing corpus of student submission
behavior, and a growing amount of past submission data to be analyzed in the
future. In our study, we hope to add to this corpus by comparing our results
with previous results, and by looking at the impact of an introduced feedback
delay on student submission behavior.
