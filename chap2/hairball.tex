\chapter{Using Static Analysis to Assist with the Post-Assessment of a
  Scratch-based \nth{6}--\nth{8} Grade Summer Camp}
\label{chap:hairball}

In this chapter, I look at the significant role that static analysis plays in
the evaluation of the overall effectiveness of our two-week Scratch-based
\nth{6}--\nth{8} grade summer camp.\footnote{The content of this chapter was
  published in SIGCSE 2013 under the title ``Hairball: Lint-inspired Static
  Analysis of Scratch
  Projects''. \url{http://dx.doi.org/10.1145/2445196.2445265}} Static analysis
is a technique for automatically analyzing computer programs to gain insights
into properties such as correctness, soundness, and simplicity. Scratch is a
building-block programming language designed for kids which allows them to
create programs in a manner similar to how they would construct physical
structures with LEGO\textregistered{}. I apply static analysis to \sprogram{s}
created by \nth{6}--\nth{8} grade students in order to assess student success
with camp assignments.

\iffull
\def\currentprefix{hairball}
\subimport{sections/}{introduction}
\subimport{sections/}{relatedwork}
\subimport{sections/}{analysis_framework}
\subimport{sections/}{plugins}
\subimport{sections/}{methodology}
\subimport{sections/}{results}
\subimport{sections/}{conclusion}
\fi
