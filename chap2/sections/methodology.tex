\section{Methodology} \locallabel{sec:methodology}
In the remainder of this chapter, we will use the term Hairball to refer to
both the Hairball framework and its set of plugins as described in
Section~\localref{sec:plugins}.

We tested Hairball on the \sprogram{s} submitted during our two-week summer
camp.  There were five assignments total, with a distribution of concept
requirements. For example, complex animation was taught toward the end of the
camp, thus instances of this concept were only present in the last two
assignments, whereas initialization was present in
all~\cite{Franklin:2013:SBO}.

We first performed a manual analysis on all fifty-eight of the submitted
\sprogram{s}.  Three members of our staff independently analyzed the first five
\sprogram{s} submitted for a given assignment using a common rubric. We
discussed any discrepancies in our scores, and after coming to a consensus, we
analyzed the remaining \sprogram{s}. Once again, any score discrepancies were
reconciled.

Hairball was then programmed to match the methodology agreed upon by the staff
members when classifying the concepts, and subsequently used to statically
analyze all of the \sprogram{s}.  When there were discrepancies between
Hairball and the manual analysis, we performed a second manual analysis to
determine which was correct, Hairball or the initial manual analysis. In
Section~\localref{sec:results}, we compare the results between Hairball and the
reconciled manual analysis using the results of the second manual analysis as a
ground truth.

Because the assignments are sensory in nature (auditory, visual), we are not
attempting to create Hairball to replace manual analysis.  Instead, we are
automating the identification of the \emph{easy} cases in order to accelerate
analysis.  The methodology is not perfect because Hairball was informed by our
manual analysis. In this regard, Hairball could be considered a fourth entity
whose answers needed to be reconciled with the group.  As the results show,
however, Hairball did an excellent job of identifying issues that all three of
our staff members missed.
