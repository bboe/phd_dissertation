\section{Introduction}
There is a movement toward both more interactive and more engaging assignments
and languages for introductory and AP computer science courses. This movement
includes the push for Python with Multimedia approaches, the various approaches
to the AP Computer Science Principles course, as well as Alice and
Scratch~\cite{Adams:2012:SLP:2157136.2157319, Forte:2004:CCC:962752.962945,
  Simon:2010:ERC:1822090.1822151, Snyder:2012:FFC:2189835.2189852,
  Cooper:2003:TOI:611892.611966, Maloney:2010:SPL:1868358.1868363}.

One drawback of assignments written in building-block programming languages
such as Alice and Scratch is that their evaluation can be more difficult than
traditional text-based programming assignments.  A common and straightforward
practice in evaluating text-based assignments is to perform functional
testing. That is, to write a script to run all submitted programs and compare
their output with solution files~\cite{Jackson:1997:GSP:268084.268210}.  More
recently, unit-testing frameworks have been employed as part of automated
assessment~\cite{Spacco:2006:EMD:1140124.1140131,
  Edwards:2003:RCS:949344.949390}.  When students are given creative freedom
with a sensory assignment --- an integral feature of languages such as Alice
and Scratch --- there is neither a text-based output file to compare to an
expected output, nor a straightforward way to perform unit-testing.  For
example, Scratch evaluation typically requires that each \sprogram{} be
individually opened and run.  Inspection of \sprogram{} code requires many
mouse clicks and navigation through a number of Scratch objects including the
\stage{} and all \emph{sprites} as well as the associated \emph{scripts} of
each.

\begin{description}
\item[stage] A single background object in Scratch that is otherwise nearly
  identical to a \emph{sprite}.
\item[sprite] An object in Scratch. Any number of \emph{sprites} can be added
  to a \sprogram{} each of which has its own set of attributes (e.g., position
  and orientation) and can be associated with any number of \emph{scripts}.
\item[script] A series of one or more executable code statements (a
  \emph{block}). Each script is associated with either a \emph{sprite} or the
  \emph{stage}.
\end{description}

To assist with assessment of \sprogram{s}, we propose a static analysis tool.
Inspired by the Scratch mascot, a cat, and the concept of lint, a static
analysis utility for C that looks for potential defects with program code, we
call our system Hairball~\cite{Johnson78lint}. We propose two roles for
Hairball:

\begin{description}
\item[formative assessment] Black and Wiliam broadly define formative
  assessment as, ``all those activities undertaken by teachers, and/or by their
  students, which provide information to be used as feedback to modify the
  teaching and learning activities in which they are
  engaged''~\cite{black1998assessment}. Inspired by lint, we envision students
  will use Hairball as a form of formative assessment by receiving feedback on
  potential problems in the \sprogram{s} they are working on.
\item[summative assessment] Summative assessment generally refers to an overall
  assessment of a course or an assignment. In our context, researchers and
  instructors can accelerate manual analysis of \sprogram{s} required for
  summative assessment by using Hairball to verify the presence and correct use
  of required computer science constructs within their students' \sprogram{s}.
\end{description}

We developed a plugin architecture so that, in Python, Hairball can be extended
and adapted for evaluation of specific assignments, and tested Hairball on
fifty-eight assignments created by \nth{6}--\nth{8} grade students during our
two-week Scratch-based summer camp in 2012.

The challenges we explore in this chapter relate to where the line should be
drawn between what Hairball can do with static analysis, and where manual
examination of the \sprogram{} is necessary.  We find that each has its own
strengths.  Hairball can quickly differentiate between \sprogram{s} that do, or
do not, contain certain targeted constructs. Hairball is also particularly
helpful for identifying instances of various constructs and implementations
that are not robust, but may not immediately cause obvious errors at runtime.
Manual analysis, however, is still needed to evaluate the overall aesthetic
effect and cohesion of a visual or auditory assignment.

We begin in Section~\localref{sec:background} by providing a background on
automated analysis in general and for Scratch in particular. We describe our
Hairball framework in Section~\localref{sec:framework}. The Hairball plugins we
developed for our analysis are described in Section~\localref{sec:plugins}. We
describe our methodology in Section~\localref{sec:methodology}, and results in
Section~\localref{sec:results}. Finally, in Section~\localref{sec:conclude} we
conclude.
