\section{Related Work}\locallabel{sec:background}

Providing automation for analyzing traditional programs is not a new concept.
ASSYST is an automated assessment system that performs end-to-end, or
input/output, type testing of
submissions~\cite{Jackson:1997:GSP:268084.268210}. Both Marmoset by Spacco et
al., and Web-CAT by Edwards, perform testing of code using student written unit
tests~\cite{Spacco:2006:EMD:1140124.1140131,
  Edwards:2003:RCS:949344.949390}. All of the aforementioned tools supplement
the feedback the student receives with code coverage analysis and feedback from
static analysis tools such as FindBugs by Cole et
al.~\cite{Cole:2006:IYS:1176617.1176667}. Douce et al. performed a more
detailed analysis of existing automated assessment
systems~\cite{Douce:2005:ATA:1163405.1163409}.  The problem with existing
assessment systems is that they are not applicable to \sprogram{s}.

Scratch is a block-based programming language from
MIT~\cite{Maloney:2010:SPL:1868358.1868363}.  \sprogram{s} consist of
two-dimensional interactive animations.  Objects, or sprites, move on the
screen as a result of user input or scripts in a \sprogram{}.  Sound and video
can also be integrated into \sprogram{s}.  Scratch was designed to allow
students to learn computer science programming while employing great creativity
in their work.  This creative freedom is one of the reasons that \sprogram{s}
are challenging to analyze.

An additional challenge in Scratch analysis is that \sprogram{s} are developed
and run within a graphical user interface.  Independent segments of code, known
as scripts, are associated with Scratch sprites, e.g., the Scratch Cat, and
tied to a triggering event.  There is no central \emph{main} point of
execution.  Instead, \sprogram{s} might begin when a parallel set of scripts
beginning with a \greenflag{} hat block are triggered.

Little prior work has looked at automated Scratch analysis.  Adams and Webster
describe using scripts and custom modifications to the Squeak source code of
Scratch to perform their quantitative analysis of \sprogram{s} from the
Imaginary Worlds summer camp~\cite{Adams:2012:SLP:2157136.2157319}.
Additionally, Burke and Kafai developed Scrape as a visualization tool to aid
humans in understanding patterns across Scratch
files~\cite{scrape-poster}. Scrape was used to assess \sprogram{s} produced in
a middle school writing workshop~\cite{Burke:2012:WWY:2157136.2157264}. Scrape
is useful in answering questions such as:
\begin{itemize}
\vspace*{-.025in}
\item How many \sprogram{s} use loops?
\vspace*{-.025in}
\item How  many loops are present in each \sprogram{}?
\vspace*{-.025in}
\item What level of nesting does the \sprogram{} use?
\vspace*{-.025in}
\end{itemize}

Hairball has two purposes.  Like Scrape, Hairball can help verify the use of
the required constructs of an assignment.  The main contribution of Hairball,
however, is the framework and set of available plugins that support more
sophisticated analysis.  We want to answer questions not just about the use of
computer science constructs, but about the competence demonstrated for
different computer science concepts. Hairball can be used to answer questions
such as:
\begin{itemize}
\vspace*{-.025in}
\item Which \sprogram{s} contain unmatched \broadcast{} and \receive{} blocks?
\vspace*{-.025in}
\item Which \sprogram{s} contain broadcast and receive events that result in
  infinite loops?
\vspace*{-.025in}
\item Which \sprogram{s} do not properly initialize the start state?
\vspace*{-.025in}
\item Which \sprogram{s} do not properly implement complex animations
  (requiring the application of timing, costume changes, motion, and loops)?
\vspace*{-.025in}
\end{itemize}
\vspace*{0.25in}
