\section{Background and Related Work}\locallabel{sec:background}

Providing automation for analyzing programs is not a new concept.  ASSYST is an
automated assessment system that performs end-to-end, or input/output, type
testing of submissions~\cite{Jackson:1997:GSP:268084.268210}. Both the Marmoset
system by Spacco et al., and the Web-CAT system, by Edwards, perform testing of
code using student written unit tests~\cite{Spacco:2006:EMD:1140124.1140131,
  Edwards:2003:RCS:949344.949390}. All of the aforementioned tools supplement
the feedback the student receives with code coverage analysis and feedback from
static analysis tools such as FindBugs by Cole et
al.~\cite{Cole:2006:IYS:1176617.1176667}. Douce et al. performed a more
detailed analysis of existing automated assessment
systems~\cite{Douce:2005:ATA:1163405.1163409}.  The problem with existing
assessment systems is that they are not applicable to Scratch programs.

Scratch is a block-based programming language from
MIT~\cite{Maloney:2010:SPL:1868358.1868363}.  Programs consist of 2-dimensional
interactive animations.  Objects, or sprites, move on the screen as a result of
user input or scripts in the program.  Sound and video can also be integrated
into Scratch programs.  Scratch was designed to allow students to learn
computer science projects while employing great creativity in their work.  This
creative freedom is one of the reasons that Scratch projects are challenging to
analyze.

An additional challenge in Scratch analysis is that Scratch programs are
developed and run within a graphical user interface (GUI).  Independent
segments of code, known as scripts, are associated with Scratch sprites
(e.g. the Scratch Cat) and tied to a triggering event.  There is no central
``main'' point of execution.  Instead, programs might begin when a parallel set
of scripts beginning with a ``when green flag clicked'' hat block are
triggered.

At least two other projects have looked at automated Scratch analysis.  Adams
and Webster describe using scripts and custom modifications to the Squeak
source code of Scratch to perform their quantitative analysis of programs from
the Imaginary Worlds summer camp~\cite{Adams:2012:SLP:2157136.2157319}.
Additionally, Burke and Kafai developed Scrape~\cite{scrape-poster} as a
visualization tool to aid humans in understanding patterns across Scratch
files. Scrape was used to assess Scratch projects produced in a middle school
writing workshop~\cite{Burke:2012:WWY:2157136.2157264}. Scrape is useful in
answering questions such as:
\begin{itemize}
\vspace*{-.025in}
\item How many programs use loops?
\vspace*{-.025in}
\item How  many loops are present in each program?
\vspace*{-.025in}
\item What level of nesting does the program use?
\vspace*{-.025in}
\end{itemize}

Hairball has two purposes.  Like Scrape, Hairball can help verify the use of
the required constructs of an assignment.  The main contribution of Hairball,
however, is the framework and set of available plugins that support more
sophisticated analysis.  We want to answer questions not just about the use of
Computer Science (CS) constructs, but about the competence demonstrated for
different CS concepts. Hairball can be used to answer questions such as:
\begin{itemize}
\vspace*{-.025in}
\item Which programs contain unmatched broadcast and receive blocks?
\vspace*{-.025in}
\item Which programs contain broadcast/receive events that result in infinite
  loops?
\vspace*{-.025in}
\item Which programs do not properly initialize the start state?
\vspace*{-.025in}
\item Which programs do not properly implement complex animations (requiring
  the application of timing, costume changes, motion, and loops)?
\vspace*{-.025in}
\end{itemize}
\vspace*{0.25in}
