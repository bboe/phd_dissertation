\section{Design Considerations}\locallabel{sec:framework}

We have two goals in designing Hairball. Our first goal is to perform analysis
on a set of \sprogram{s} automatically. Without automated analysis, inspection
and execution requires opening each \sprogram{} manually. This manual process
is time-consuming and error-prone.  Our second goal is that Hairball is easily
extendable so that new Scratch analysis plugins can be created with only a
basic amount of Python experience, and anyone can make use of available
plugins.


\subsection{Plugin Architecture}
We used the object-oriented features of Python to develop a base class from
which Hairball plugins can be derived. Python was chosen due to the authors'
experience with Python and its increased adoption in introductory computer
science classes. However, Python was mainly chosen due to the open source
Python package \emph{Kurt} that provides simple access to all the elements
contained within a \sprogram{}, i.e., the images, sounds, stages (backgrounds),
sprites and most importantly the scripts~\cite{Radvan:2012:Online}.\footnote{As
  part of our work, we made a few contributions that are now a core part of the
  \emph{Kurt} Python package.}

Implementing a Hairball plugin simply requires extending the base class and
overloading a single method. The method's sole parameter is a handle to the
\sprogram{} (from \emph{Kurt}) and the method should return a dictionary
containing the results of the desired static analysis. In principle, any type
of static analysis of a \sprogram{} that can be described algorithmically can
be implemented as a Hairball plugin in a straightforward manner by anyone with
basic Python programming skills. The following code provides an example of a
simple Hairball plugin that counts the number of times each Scratch block is
used in a \sprogram{}.

\begin{verbatim}
class BlockCounts(HairballPlugin):
  def analyze(self, scratch):
    blocks = Counter()
    for block, _, _ in iter_blocks(scratch):
      blocks.update({block: 1})
    return blocks
\end{verbatim}
