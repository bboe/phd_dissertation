\section{Hairball Plugins} \locallabel{sec:plugins}
In this section we describe four Hairball plugins written to perform Scratch
static analysis.  The plugins were designed to analyze projects submitted as
part of our two-week interdisciplinary Animal Tlatoque summer
camp~\cite{Franklin:2013:SBO}.  Notice that some of the traditional topics,
i.e., variables and conditionals, are not represented, and that loops are
represented in a very specific way (defined as animation). The plugins target
the CS concepts used in the camp's cumulative project, an interactive movie
about an animal. For this project, students were to demonstrate state
initialization, use of broadcast and receive blocks, synchronization between
say and sound blocks, and creation of complex animation. While these plugins
were developed for our summer camp, each provides valuable feedback that is
generally useful both as a lint-like tool for individual developers of Scratch
programs and for others who are tasked with analyzing numerous Scratch
programs.

Each Hairball plugin for the camp strives to evaluate whether, or to what
extent, the program demonstrated competence in an area. These plugins attempt
to discover instances of the aforementioned concepts contained within a Scratch
program and label each instance as {\bf correct}, {\bf semantically incorrect},
{\bf incorrect}, or {\bf incomplete}. Instances labeled ``correct'' should
indicate that the concept was implemented correctly. Instances labeled
``semantically incorrect'' should indicate that the concept was implemented in
a way that may not always work when executed. Intuitively, instances labeled
``incorrect'' should indicate the concept was implemented incorrectly. Finally,
instances labeled ``incomplete'' should indicate that only a subset of the
required blocks for a concept was discovered. A single program may contain
multiple instances of a concept distributed across any or all of the
labels. Ideally instances labeled ``correct'' should not require manual
analysis, whereas instances with any other label should be inspected manually.


\subsubsection*{Initial State}

\begin{table}
\centering
\begin{tabular}{|c|c|c|} \hline
Category & Relative&Absolute\\ \hline \hline
Costume& next costume & switch to costume x\\ \hline
Visibility& & show/hide\\ \hline
Orientation&turn clockwise x degrees&point in direction x\\ \hline
Position&move x steps, go to x,y,&go to x,y\\
&glide z secs to x,y, etc.&go to x,y\\ \hline
Size&change size by x\% & set size to x\%\\ \hline
Background&next background & switch to background x\\ \hline
\end{tabular}
\caption{Five categories of initial state, along with example relative and
  absolute modification blocks}
\locallabel{table:initialstate}
\end{table}


In any program, correctly setting the initial state is important.  In Scratch
programs, the significance is different.  Scratch programs are comprised of
animations, and in the runtime environment, they may run from start to finish
and be restarted again.  Alternatively, they may be stopped in the middle and
restarted again.  We want to determine statically whether the code runs the
same way in these two events.  For reasons described below, the environment is
different than in traditional programming.

The first problem is where to start the analysis.  In traditional programs,
execution starts at ``main''.  Scratch programs have no such globally defined
starting point.  We taught our students to start their programs using the green
flag button, so the starting point for our evaluation is the ``when green flag
clicked'' block.

The most complex problem, and the problem that introduces the possibility of
error into our analysis, is that sprites are placed on the stage during
implementation thus giving them an implicit set of attributes, which we will
refer to as the base attributes. Explicit initialization for a particular
attribute, e.g., position or orientation, is only required when one of the
program's scripts modifies the attribute. Thus, the challenge is distinguishing
segments of scripts that perform initialization from those that perform general
modification. To discover instances of initialization, we first determine the
set of blocks that can be considered initialization blocks and then we restrict
the location such blocks can appear in a script. We call this location the
``initialization zone''

Attribute modifying Scratch blocks can be labeled as ``relative'' or
``absolute''. Relative Scratch blocks alter the attribute based upon its
current value whereas absolute Scratch blocks directly set the attribute. As
such only absolute blocks can be considered initialization
blocks. Table~\localref{table:initialstate} shows our categorization for a
subset of attribute modifying Scratch blocks.

For an absolute block to be considered an initialization block, it must appear
in the initialization zone. We define the initialization zone only for scripts
beginning with a ``when green flag clicked'' block. The initialization zone
begins at the start of the script and ends when either a relative block or a
broadcast block is encountered. We take a conservative approach when
encountering blocks contained within loops or conditionals---absolute blocks
are ignored due to the possibility that the block is not executed, and relative
blocks continue to signify the end of the initialization zone due to the
possibility that the block is executed.

The initialization plugin considers a modified attribute of a sprite as
correctly initialized when an absolute block for the same attribute exists in
the initialization zone. Instances are labeled as incorrect
otherwise. Non-modified attributes are ignored. Finally, despite this plugin's
ability to detect unnecessary initialization, we did not include it as part of
our analysis.


\subsubsection*{Say and Sound Synchronization}
Synchronization between a speech bubble (say block) and sound file (play sound
block) is not straightforward in Scratch.  The desired behavior is that, at the
same time, a speech bubble appears with the message, and a sound file plays of
a voice speaking the message.  When the sound is complete, the speech bubble
disappears.

Achieving this effect is complicated by the timing semantics of the two forms
of the say block, and the two forms of the play sound block in Scratch.  One
form of the say block places the speech bubble on the screen indefinitely
(until replaced by another say block, or ``erased'' with an empty say block),
while the other puts a speech bubble on the screen for $n$ seconds (and, as a
side-effect, delays execution of the script for $n$ seconds.)  Similarly, there
are two forms of the block for playing a sound clip: ``play sound until done''
plays the sound synchronously, while ``play sound'' plays the sound
asynchronously.

There are two methods to produce the desired effect.  The first is to
asynchronously play the sound via the ``play sound block'' followed by a ``say
for'' block with duration equal to the elapsed time of the sound.
Unfortunately, the timing must be manually determined and needs to be updated
whenever the sound file changes.  The second, is to use a ``say'' block to
display the message, followed by a ``play sound until done'' block, ending with
an empty ``say'' block to clear the previous speech bubble. The campers were
taught the latter method as the correct approach.

This plugin detects instances of this concept by looking for sequential say and
sound blocks and verifies the instances are implemented using the appropriate
method.  A correct instance contains the previously described three blocks in
the proper order.  Instances following the method requiring manual timing are
labeled semantically incorrect. Instances that have both say and sound blocks
but do not match either of these methods are labeled incorrect, and isolated
uses of say or sound blocks are labeled incomplete.


\subsubsection*{Broadcast and Receive}
One use of Scratch's broadcast blocks is to trigger the execution of other
sprites' scripts beginning with the appropriate receive block. We taught our
campers the broadcast and receive concept in the context of two animal sprites
conversing, where each sprite would signal the other's turn via broadcasting an
event. In the camp's cumulative project, campers demonstrated an understanding
of the broadcast and receive concept by triggering scene changes in their
interactive movie.

The broadcast and receive plugin verifies that for each broadcast or receive
event, there is a broadcast block and at least one corresponding receive
block. Such instances are labeled correct. All instances with a broadcast block
appearing in the same script with another instance's broadcast block are
labeled as semantically incorrect. All other instances are labeled
incomplete. Note that this plugin does not use the incorrect label.

\subsubsection*{Complex Animation}
We have a very specific definition of the term {\em complex animation} for
purposes of assessment.  We use this term to refer to animation involving
integration of costumes, motion, timing, and repetition control structures such
as loops. This definition of complex animation is to distinguish from, for
example, the ``glide to'' block built into Scratch.  One example of complex
animation is realistic motion of sprites that represent people and
animals. E.g., people walking, birds flying and snakes slithering.  Creating
such animations requires the correct integration of several Scratch and CS
concepts.  For example, creating an animation sequence where a sprite spins
around requires integration of loops, rotation, and timing.

A necessary component of a complex animation instance is either rotation
blocks, or motion blocks paired with costume change blocks. We define a complex
animation instance as either a loop containing these necessary components or a
sequence of these necessary components, since a sequence can be considered an
unrolled loop. In order to be labeled correct, an instance must also make use
of a Scratch block that introduces a delay; otherwise the instance is labeled
semantically incorrect. The plugin additionally labels instances that use
sequences instead of loops as semantically incorrect because the student did
not demonstrate competence in the CS concept of loops. Finally, if the program
is missing any critical element, e.g., repetition, it is labeled incomplete.
